\documentclass{article}

\usepackage{geometry}
\usepackage{makecell}
\usepackage{array}
\usepackage{multicol}
\usepackage{setspace}
\usepackage{changepage}
\usepackage{booktabs}
\usepackage[explicit]{titlesec}
\usepackage{hyperref}
\usepackage{graphicx}
\usepackage{cprotect}
\usepackage{float}
\newcolumntype{?}{!{\vrule width 1pt}}
\newcommand{\paragraphlb}[1]{\paragraph{#1}\mbox{}\\}
\renewcommand{\contentsname}{Inhaltsverzeichnis:}
\renewcommand\theadalign{tl}
\setstretch{1.10}
\setlength{\parindent}{0pt}

\titleformat{\section}
  {\normalfont\Large\bfseries}{\thesection}{1em}{\hyperlink{sec-\thesection}{#1}
\addtocontents{toc}{\protect\hypertarget{sec-\thesection}{}}}
\titleformat{name=\section,numberless}
  {\normalfont\Large\bfseries}{}{0pt}{#1}

\titleformat{\subsection}
  {\normalfont\large\bfseries}{\thesubsection}{1em}{\hyperlink{subsec-\thesubsection}{#1}
\addtocontents{toc}{\protect\hypertarget{subsec-\thesubsection}{}}}
\titleformat{name=\subsection,numberless}
  {\normalfont\large\bfseries}{\thesubsection}{0pt}{#1}

\hypersetup{
    colorlinks,
    citecolor=black,
    filecolor=black,
    linkcolor=black,
    urlcolor=black
}

\geometry{top=12mm, left=1cm, right=2cm}
\title{\vspace{-1cm}Präsentationstechnik}
\author{Andreas Hofer}

\begin{document}
	\maketitle
	\tableofcontents
	\section{Aufbaue einer Präsentation}
	Eine Präsentation entsteht in vier groben Schritten:
	\begin{itemize}
		\item{Planung}
		\begin{itemize}
			\item{Wer präsentiert wann, wo, was und wie?}
			\begin{itemize}
				\item{Es kann sehr wichtig sein, wann man präsentiert, da ein sehr früher Zeitpunkt und ein sehr später Zeitpunkt unterschiedliche Strategien bedürfen.}
				\item{Man sollte wissen wo man präsentiert, da man so weiß welches Equipment zur Verfügung steht, wie viele Leute wahrscheinlich auftauchen und wer das Publikum sein wird.}
			\end{itemize}
		\end{itemize}
		\item{Vorbereitung}
		\item{Durchführung}
		\item{Nachbereitung}
		\begin{itemize}
			\item{Überlegungen der getätigten Präsentation um heruaszufinden was man gut oder schlecht gemacht hat.}
		\end{itemize}
	\end{itemize}
	\subsection{Zielformulierung}
	In der Planung ist es wichtig das Ziel festzulegen, i.e. festzustellen, was man mit seiner Präsentation eigentlich erreichen will
	\subsubsection{Zielgruppe}
	Das beinhaltet auch die Definition der Zielgruppe. Dabei muss man überlegen, was die Alterstruktur des Publikums sein wird, oder welches Vorwissen erwartet werden kann. Gibt es eventuell Werte oder Einstellungen welche man beim Publikum























  
\end{document}