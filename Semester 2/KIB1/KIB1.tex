\documentclass{article}

\usepackage{geometry}
\usepackage{makecell}
\usepackage{array}
\usepackage{multicol}
\usepackage{setspace}
\usepackage{changepage}
\usepackage{booktabs}
\usepackage{graphicx}
\usepackage[explicit]{titlesec}
\usepackage{hyperref}
\usepackage{cprotect}
\usepackage{float}
\newcolumntype{?}{!{\vrule width 1pt}}
\newcommand{\paragraphlb}[1]{\paragraph{#1}\mbox{}\\}
\renewcommand{\contentsname}{Inhaltsverzeichnis:}
\renewcommand\theadalign{tl}
\setstretch{1.10}
\setlength{\parindent}{0pt}

\titleformat{\section}
  {\normalfont\Large\bfseries}{\thesection}{1em}{\hyperlink{sec-\thesection}{#1}
\addtocontents{toc}{\protect\hypertarget{sec-\thesection}{}}}
\titleformat{name=\section,numberless}
  {\normalfont\Large\bfseries}{}{0pt}{#1}

\titleformat{\subsection}
  {\normalfont\large\bfseries}{\thesubsection}{1em}{\hyperlink{subsec-\thesubsection}{#1}
\addtocontents{toc}{\protect\hypertarget{subsec-\thesubsection}{}}}
\titleformat{name=\subsection,numberless}
  {\normalfont\large\bfseries}{\thesubsection}{0pt}{#1}

\hypersetup{
    colorlinks,
    citecolor=black,
    filecolor=black,
    linkcolor=black,
    urlcolor=black
}

\geometry{top=12mm, left=1cm, right=2cm}
\title{\vspace{-1cm}Konfigurationsmanagement 1}
\author{Andreas Hofer}

\begin{document}
	\maketitle
	\tableofcontents
	\newpage
	Configuration management is based on the concept of continuous integration, a process in which software is improved upon over a longer period of time in order to improve, further and enhance its functionality as well as documenting the process. In addition it also concerns itself with delivering executables in many different environments. \\
	This lecture is structured in six sections:
	\begin{enumerate}
		\item{Documentation}
		\begin{itemize}
			\item{Informing the user or other maintainers about the program's functions or requirements}
		\end{itemize}
		\item{Project Structure and Layout}
		\begin{itemize}
			\item{[TODO] Project Structure}
		\end{itemize}
		\item{Version Management}
		\begin{itemize}
			\item{Managing different iterations of projects as well as being able to recover from}
		\end{itemize}
		\item{Building}
		\begin{itemize}
			\item{The automatic creation of executables and figuring out issues with the build process}
		\end{itemize}
		\item{Testing}
		\begin{itemize}
			\item{Automatically testing components in order to figure out problems quickly}
		\end{itemize}
		\item{Releasing}
		\begin{itemize}
			\item{Releasing }
		\end{itemize}
	\end{enumerate}
	\section{Documentation}
	Documentation serves to inform the user or another programmer about the functions, how to maintain or change the poject, or how to use it. One of the most common ways to implement that is the so-called README file. \\
	\subsection{README}
	READMEs are usually very small text files added in the top-most folder of the application which briefly explains how to use it. It is generally a good idea to create the README \textit{before} starting with the project. A popular format to write them in is \textit{Markdown}, a markup language. \\
	Markup Languages are a family of languages which are used to format text in a specific way. HTML (Hypertext Markup Language) is one of these languages but tends to be difficult to use quickly, when all you want is to write how to use a program. Latex too is a markup language but isn't too handy without compiling it to a pdf, at which point it cannot be changed anymore. \\
	Because of that Markdown (MD) is an ideal format to write documentation in, since it only uses minimal formatting to \textit{italise} or \textbf{bold} text.\\
	\subsubsection{Markdown Syntax}
	Some formatting rules for Markdown are:
	\begin{itemize}
		\item{\textit{italic text} is created using *single asterisks*}
		\item{\textbf{bold text} is created using a **double asterisk**}
		\item{Headers can be created in two different ways:}
		\begin{itemize}
			\item{Either by adding pound signs to the left of the title. The higher the number of pound signs, the lower the header. \verb|# Title 1 / ## Title 2|}
			\item{Alternatively the first and second header can be created using either = or - in a new line beneath it respectively.}
			\item{\texttt{Title 1}}
			\item[]{\texttt{======}}
			\item{\texttt{Title 2}}
			\item[]{\texttt{------}}
		\end{itemize}
		\item{Headers are signified by using the pound sign \#, while depending on the number of consecutive signs to go further down in the hierarchy}
		\item{In order to create lists, there are several options}
		\begin{itemize}
			\item{For an unordered list, add one asterisk to the left of the *item}
			\item{For an ordered list, add the number to the left of the 1. item}
		\end{itemize}
		\item{In order to use any of these special characters within the text you can add a backslash before it (Escaping it) \\*}
	\end{itemize}

	\newpage
	\subsection{Cheatsheet}
	\begin{tabular}{| l | l | l |}
		\toprule
		Name & Markdown & Result \\ \midrule
		italic & *italic* & \textit{italic} \\ \hline
		bold & **bold** & \textbf{bold} \\ \hline
		
		\bottomrule
	\end{tabular}
	\newpage
	\section{Project Structure}
	With increasing project size, the amount of loose files can become hard to keep track of. To make that easier, it is a good idea to group files based on their purpose within the project. IDEs tend to automatically create grouping folders so it is not all just lying in one folder.
	\subsection{Maven}
	One system to help with this is \textbf{Maven} which helps manage Java projects across its lifecycle. It provides a uniform build system by standardising the process through the so-called \textit{pom.xml} (Project Object Model). It includes basic information about the project as well as build and environment settings. It for example automatically downloads dependencies listed in it so the user doesn't have to manage it. This may seem overkill for smaller projects but it becomes essential for larger projects. Often different components require different versions of libraries and management files help maintain these dependencies. In addition it also enables the maintainer to share the pom.xml file and ideally immediately have the build system set up the environment. Another feature is automated testing, where the system can automate running and evaluating all available unit tests. \\
	Maven achieves this structure by using predefined folders for specific project components so it is uniform between all Maven projects. For example it fully separates build and test environments, grouping these in different folders.
	\section{Versioning}
	Versioning helps with maintaining a clear structure for logging changes on the project. So whenever anything is added, removed or changed, the versioning system will log it. It also helps to return to a previous version should an error have emerged in a later version. This is not too relevant for small projects but large enterprise systems tend to have thousands over thousands lines of code with many hundred classes all interacting with each other. Maybe some of the maintainers are no longer part of the organisation or working somewhere else but with versioning you can see who made which changes. In addition it prevents careless changes to production code by requiring Pull Requests (Requests to change the repository code) to be confirmed by another person. \\
	Versioning systems can also help more efficiently comparing different versions by visually adjusting the code to show where something was added or removed. These versions can also be commented to add context to a change outside of comments within the code.
	\subsection{Types of Versioning}
	There are three kinds of versioning systems:
	\begin{itemize}
		\item{Local}
		\begin{itemize}
			\item{All versioning is exclusively stored on your computer preventing access from other people}
		\end{itemize}
		\item{Centralised}
		\begin{itemize}
			\item{Code is saved on a single server which multiple people can access and change}
		\end{itemize}
		\item{Distributed}
		\begin{itemize}
			\item{Each server and user has a full copy of the code}
			\item{Any changes are distributed to all users}
		\end{itemize}
	\end{itemize}
	\subsubsection{Local}
	Local versioning software is extremely easy to set up since all of the changes are only ever added to your own hard drive. It is very fast, since you have direct physical access to the medium and you also do not need to worry about security. \\
	As a disadvantage it is relatively difficult to cooperate with other people unless they come to your location. It is also not save from spontaneous data loss like a hard drive malfunction.
	\subsubsection{Centralised}
	In a centralised solution all the versions are only saved on a central server. Users tend to only use the version they are currently working on without having access to the entire version history. These systems allow for collaboration while being relatively easy to maintain since they only require a single server. But this single server is also a single point of failure and data loss can lead to a loss of most of the data. Since the code is only saved on the server, all access has to go through it as well, making access relatively slow.
	\subsubsection{Distributed}
	When each server and user maintains a full copy of the versions, a user has a lot more flexibility when making changes. Since everybody has all of the data you can make changes while offline and only synchronise when needed, upon which everybody receives the new version. \\
	This system is very robust since it has no single point of failure and upon data loss, there is a good chance that another user can use their own copy as a backup. But it is relatively data hungry sinc each user has everything available.
	\section{Git}
	The most commonly used versioning tool today is Git. It was developed by Linus Torvalds, creator of Linux, in order to make it easier to collaborate in its maintenance. Git changes the old approach of versioning systems to only save the changes from the previous version and instead saves a new version of a branch in its entirety. This change was made because large projects with hundreds of thousands of commits became painfully slow to figure out the newest version since it first had to calculate it based on all previous changes. This also made it possible to more easily branch away from the main version, which before that was slow and expensive to use. Git also calculates a checksum of each commit in order to preserve the integrity of this version and making data manipulation nearly impossible.
	\subsection{Stages}
	Within Git files can be in three stages:
	\begin{itemize}
		\item{Modified}
		\begin{itemize}
			\item{An already existing file had been modified}
		\end{itemize}
		\item{Staged}
		\begin{itemize}
			\item{A new or modified file is added to be committed}
		\end{itemize}
		\item{Committed}
		\begin{itemize}
			\item{Files have been committed and added to the versioning}
		\end{itemize}
	\end{itemize}
	\subsection{Recording Changes}
	Git additionally tracks the states files are in.
	\begin{itemize}
		\item{Untracked}
		\begin{itemize}
			\item{Files which have not (yet) been added to the versioning.}
			\item{Can be added by using \texttt{git add}}
		\end{itemize}
		\item{Unmodified}
		\begin{itemize}
			\item{Files which have been committed are changed to unmodified}
		\end{itemize}
		\item{Modified}
		\begin{itemize}
			\item{Files which have been changed are saved as modified until they are committed again}
		\end{itemize}
		\item{Staged}
		\begin{itemize}
			\item{Modified or new files which are set to be committed}
		\end{itemize}
	\end{itemize}
	\begin{figure}[H]
	\centering
	\includegraphics[scale=0.4]{Bilder/git_states.png}
	\caption{Different states of files in Git}
	\end{figure}
	
	\subsection{Commands}
	Git by default is a Command Line Interface (CLI) so it also uses commands to access its features. Some of its most important commands are:
	\subsubsection{\texttt{init}}
	\texttt{Init} initialises a new git repository in the current folder and automatically adds the main branch.
	\texttt{git init -> Creates new Git repository}
	\subsubsection{\texttt{status}}
	\texttt{status} shows the current state of the repository, informing you which branch you are currently on and all changes currently set to be committed. (Achieved through \texttt{add}) \\
	\texttt{git status -> Prints the branch status}
	\subsubsection{\texttt{add}}
	\texttt{Add} adds all specified files found in the folder. It supports batching commands so you can use either '*' or '.' to add all available files. \\
	\texttt{git add . -> Adds all files that have been changed or added}
	\subsubsection{\texttt{commit}}
	\texttt{commit} confirms all currently staged changes. A commit message is required, though it can be left blank. If commit is used without anything else the command line will automatically enter a text editor prompting you for a commit message. Alternatively the option '-m' can be used together with a message enclosed in double quotes to skip this. \\
	\texttt{git commit -m "Commit message 1" -> Commits staged changes with the message 'Commit message 1'}
	\subsection{Remote}
	All of the previous actions have happened locally on a personal pc. In Git this is usually known as 'Local'. On the other hand the server you are synchronising to is called 'Remote'. Because Git is open-source there are multiple solutions to using a Remote server, the most popular of which is Github.
	\subsubsection{Remote Commands}
	In order to add a remote repository and synchronise with it, there are multiple commands. These are:
	\begin{itemize}
		\item{\texttt{push}}
		\begin{itemize}
			\item{Pushes all commits stored locally which aren't available on the server to the repository. If the repository is ahead of the local version (Has changes that aren't available locally) you might have to pull first.}
			\item{\texttt{git pull -> Pulls all changes from remote}}
		\end{itemize}
		\item{\texttt{pull}}
		\begin{itemize}
			\item{Pulls all commits not stored locally from the remote repository. If there are conflicts from this you might have to make changes manually and confirm.}
			\item{\texttt{git push -> Pushes all changes to remote}}
		\end{itemize}
		\item{\texttt{clone}}
		\begin{itemize}
			\item{Creates a copy of the remote repository and initialises it locally with all previous changes, adding the remote server as the remote location}
			\item{\texttt{git clone <url> -> Pulls a new repository from <url>}}
		\end{itemize}
		\item{\texttt{remote add}}
		\begin{itemize}
			\item{As an alternative to \texttt{clone} you can use \texttt{remote add} to use} 
		\end{itemize}
		\item{\texttt{push}}
		\begin{itemize}
			\item{Uploads all local changes to the remote repository}
		\end{itemize}
		\item{\texttt{fetch}}
		\begin{itemize}
			\item{Downloads all remote changes but without merging them. You can merge them manually merge them after}
		\end{itemize}
		\item{\texttt{pull}}
		\begin{itemize}
			\item{Fetches remote changes and merges them automatically (if possible)}
		\end{itemize}
		\item{\texttt{branch}}
		\begin{itemize}
			\item{Branch only tracks local branches. In order to create a local version of a remote branch you need to create it.}
			\item{You can use the \texttt{-r} option to show all remote repositories and \texttt{-a} to show all (Local and Remote)}
		\end{itemize}
		\item{\texttt{diff}}
		\begin{itemize}
			\item{Show current and stages changes and compares it to the last commit}
		\end{itemize}
	\end{itemize}
	\subsection{Pull Request}
	Pull Requests are a way of collaboration between people. It lets you push to a repository but has it not merge immediately. Instead another user (with the right privileges) has to review the new code and only allow the merge if it is acceptable.
	\subsection{Special Git files}
	Git by default saves every file within the tracked folder. Some files are only relevant locally and somebody who downloads a copy of the project will not be interested in them. Examples of these are executable files like any .exe files or large files like images or videos. Other files you do not want are configuration files generated by the IDE or other build tools. You can tell Git to ignore certain files by adding a \texttt{.gitignore} file and adding files to be ignored there.
	\subsubsection{\texttt{.gitignore}}
	This file will specify which files to not consider while adding data to the versioning. The .gitignore file also uses batching so you can use \texttt{*.pdf} to ignore all pdf files in the current folder. To ignore all pdf files in all subfolders as well you need to write \texttt{**/*.pdf}. Alternatively you can also write out the entire path for a file to only ignore it. \texttt{src/v02/source.java} will for example only ignore the source.java file. Some commands are:
	\begin{itemize}
		\item{\texttt{*.a}}
		\begin{itemize}
			\item{Will ignore all files of type a}
		\end{itemize}
		\item{\texttt{!lib.a}}
		\begin{itemize}
			\item{If added after a more general ignore, the ! will tell Git to not ignore this specific file. The gitignore file is read sequentially so you should first add the most general commands and then add more specific ones.}
		\end{itemize}
	\end{itemize}
	\subsubsection{\texttt{.gitkeep}}
	Git by default does not save empty folders. In order to tell Git to still add this folder to the versioning, you can add a \texttt{.gitkeep} file to the folder. Git will then save the empt folder within the versioning.
	\subsection{Branching}
	Branches enable multiple people to work on different projects on the same application simultaneously. A branch creates a copy of the main branch and allows changes to this copy without affecting the main branch. This can be very useful if multiple features are being worked on at the same time, while requiring the main branch to still be fully operational. It should be noted that branching does not create a new file in and itself. Only if files are changed within that branch are these files saved again. This makes branching very efficient as previously creating a new branch would create a copy of all relevant files. \\
	Branches are used widely when creating new features, fixing bugs or testing new changes.
	\subsubsection{Branch Commands}
	In order to create a branch three commands are necessary:
	\begin{itemize}
		\item{\texttt{branch}}
		\begin{itemize}
			\item{Creates a new branch of a given name. This does not change the file structure but only creates a pointer}
			\item{\texttt{git branch feature -> Creates the branch feature}}
			\item{Alternatively deletes an existing branch with \texttt{-D}}
			\item{\texttt{git branch -D feature -> Deletes the feature branch}}
		\end{itemize}
		\item{\texttt{checkout}}
		\begin{itemize}
			\item{Switches to an already existing branch.}
			\item{\texttt{git checkout feature -> Switches to the feature branch}}
			\item{Alternatively you can add the \texttt{-b} option to create a new branch and switch to it immediately}
			\item{\texttt{git checkout -b feature -> Creates a branch called feature and switches to it}}
		\end{itemize}
		\item{\texttt{merge}}
		\begin{itemize}
			\item{Combines two branches with independent commit histories}
			\item{This can be done in two ways:}
			\item{Fast-Forward}
			\begin{itemize}
				\item{Combines two branches which have a linear path. Because of the linear path, there can be no merge conflicts and the pointer of the main branch is simply moved to the latest branch to be merged.}
			\end{itemize}
			\item{Three-Way-Merge}
			\begin{itemize}
				\item{When two branches do not have a linear history (Because the main branch has moved on while the branch existed), Git needs to perform a Three-Way-Merge to combine them. This can work automatically but can cause a merge conflict which requires manual review.}
				\item{Git can decide automatically if one change is the better one since it compares the two new versions with the base version. If only one of them has changed, it will pick that one.}
				\item{A merge conflict can arise if the same line has been modified in both versions.}
			\end{itemize}
		\end{itemize}
	\end{itemize}
	\section{Building}
	\subsection{.jar}
	In Java a Jar file is one of the key concepts in Java. It bundles multiple source files into a single archive including associated resources like images or sounds. Ultimately it is a zip archive with a different file extension, which lets you unpack the file again if you were to rename and unzip it. \\
	Another included file within the jar is the manifest file. It describes metadata about this jar file like:
	\begin{itemize}
		\item{Specifying the entry point}
		\begin{itemize}
			\item{Specifies the \texttt{Main-Class} property which is run when executed. This must include the full package name plus the class name}
			\item{You can specify a jar file to be runnable by including a manifest file with the \texttt{Main-Class} supplied by using the \texttt{cfm} option}
		\end{itemize}
		\item{Hashsum signing}
		\item{Specifying the version}
		\item{Sealing the file}
		\begin{itemize}
			\item{Stops foreign packages from being executed}
		\end{itemize}
	\end{itemize}
	Jar files can exist in two configurations: runnable or used as an external library. Runnable is executed directly (and requires a \texttt{Main-class} manifest header) while external libraries are only used as parts of other applications.
	\subsection{Compiling manually}
	While Maven can be useful to automate the compilation, it is still good to know how to run it manually: \\
	\texttt{javac <path to java file> -d <target directory> -> Compiles the java file into the target directory} \\
	Like in all CLI programs you can use batching to compile many at once by using an asterisk \texttt{*} instead of a file name. \\
	In order to subsequently run the compiled file you need to use \texttt{java} instead of \texttt{javac} and define the full package name: \texttt{java -cp <folder where class files are located> <full package name> <arguments> -> The package name including the class name must be added in order to run the file} \\
	In order to create a jar package you need to use the command line argument \texttt{cvfe}: \texttt{jar cvfe <jar file> <package name> -C <class path> . -> The full stop at the end must be added or it won't work}. \\
	Running the jar file is done by calling the regular java command: \texttt{java -jar <jar file> <command line argument>}
	\section{Maven}
	Up until now IntelliJ has likely automatically run without much hassle when pressing the play button. That's because the IDE uses a proprietary build system. An alternative, open-source, build system is Maven. Maven as a build system can manage the entire product lifecycle from managing dependencies over compiling the source files to run unit tests as well as deploy the finished product. This process is called the 'Build Process'. 
	\subsection{Automation}
	This entire process can be automated by using Maven so it doesn't have to be repeated every single iteration. This ideally works if the process is CRISP:
	\begin{itemize}
		\item{C}
		\item{Repeatable}
		\begin{itemize}
			\item{That on every build process the same result is created}
		\end{itemize}
		\item{Informative}
		\begin{itemize}
			\item{Get feedback from the build process to figure out issues easily}
		\end{itemize}
		\item{Schedulable}
		\begin{itemize}
			\item{Be able to have builds be executed on a set schedule since it might take a long time to build. Also ensures that unit tests are run regularly}
		\end{itemize}
		\item{Portable}
		\begin{itemize}
			\item{The build process should be platform agnostic and run on any system without requiring specific setup}
		\end{itemize}
	\end{itemize}
	Maven includes all of the configured files like project configurations, the source code, the test code, included libraries and documentation to create a finished jar file. The process includes:
	\begin{itemize}
		\item{Building the project}
		\item{Providing a uniform build system}
		\item{Providing project information}
		\begin{itemize}
			\item{The \texttt{pom.xml} (Project Object Model) file provides a uniform way to configure Maven projects.}
		\end{itemize}
		\item{Providing development best practice guidelines}
		\begin{itemize}
			\item{Because Maven has a default structure it promotes best practices for folder structure and naming conventions.}
		\end{itemize}
		\item{Allow transparent migration to new features}
		\begin{itemize}
			\item{Due to specific versioning for dependencies backwards compatibility is always maintained and also ensures to make migration to new versions as easy as possible}
		\end{itemize}
	\end{itemize}
	\subsection{pom.xml}
 	The pom.xml is standardised across Maven projects to ensure it does not take a lot of effort to understand the structure of a different project. With the ability to include dependencies it is easy to include plugins which provide the functionality needed. \\
 	The pom file includes certain aspects of the project:
 	\begin{itemize}
 		\item{The root folder of the project}
 		\item{dependencies}
 		\begin{itemize}
 			\item{A list of all libraries that need to be included}
 		\end{itemize}
 		\item{properties}
 		\begin{itemize}
 			\item{Defines common and reusable properties forthe projec}
 		\end{itemize}
 		\item{build}
 		\begin{itemize}
 			\item{Defines build configurations}
 		\end{itemize}
 	\end{itemize}
 	The useful part about the dependencies in the pom file is, that it automatically downloads all of the specified dependencies and versions them correctly. Since it creates a local copy of the libraries you only need to be connected to the internet while running Maven the first time and can afterwards use it offline as well. The repositories are saved in the User folder under a hidden folder called \texttt{.m2}
 	\subsection{Additional Phases}
 	In addition to the already mentioned phases of a build process, there's some more miscellaneous phases:
 	\subsubsection{Validate}
 	Ensures that all necessary information within the project is correct.
 	\subsubsection{clean}
 	Removes all generated files and folders. This is useful if you have deleted some dependencies and want to ensure it is not part of the build files anymore.
 	\section{Failure}
 	When talking about failure, you must distinguish between different tiers of failures. An error or a mistake is a human action that leads to an incorrect result (Like foregetting a semicolon). An error can potentially lead to a defect, which is a deficiency in a work product which does not meet its requirements or specifications. This defect can potentially lead to a failure which is an event in which a component in a system does nto perform within specified limits.
 	\subsection{Identifying mistakes}
 	When a failure occurs, you must reverse the error process by first identifying the failure itself, identifying the defect and then identifying the error and fixing the mistake.
 	\subsection{Debugging}
 	A way to simplify this process is testing and debugging. There are seven steps to debugging which can be abbreviated by TRAFFIC:
 	\begin{itemize}
 		\item{Track the problem}
 		\item{Reproduce it}
 		\item{Automate the testing process}
 		\item{Find infection origins}
 		\item{Focus on most likely origins}
 		\item{Isolate the infection chain}
 		\item{Correct the error behind the defect and rerun all tests (Maybe the fix has caused further errors)}
 	\end{itemize}
 	Even if you think you have fixed the mistake, make sure to only push well working code to production to not cause more issues than the fix should fix.
 	\subsection{Tiers of Testing}
 	There are four tiers of tests, which range from testing specific components in a vaccuum to testing the program within its UI:
 	\begin{itemize}
 		\item{Unit Tests}
 		\begin{itemize}
 			\item{Tests only methods or modules}
 		\end{itemize}
 		\item{Integration Tests}
 		\begin{itemize}
 			\item{Tests the execution in the final environment}
 		\end{itemize}
 		\item{End to End Tests}
 		\begin{itemize}
 			\item{Tests the product within its ecosystem}
 		\end{itemize}
 		\item{UI Test}
 		\begin{itemize}
 			\item{Tests the product within the finished UI}
 		\end{itemize}
 	\end{itemize}
 	In general the later a bug is discovered the more effort it takes to fix it. Fixing a bug while implementing it takes a few minutes but fixing a bug in a product that has already been deployed can take a very long time and require considerable effort.
 	\subsection{Unit Tests}
 	We will mainly focus on Unit Tests, which does not test the product itself but only the specific function of a single method. Ideally every production code should be tested. One metric is test coverage which specifies the percentage of the code that is covered by unit tests. \\
 	One Unit Test consists of four phases, of which not all are necessarily needed but useful.
 	\begin{itemize}
 		\item{Setup}
 		\begin{itemize}
 			\item{Sets up the environment in order to test the system}
 		\end{itemize}
 		\item{Exercise}
 		\begin{itemize}
 			\item{The system is executed with the specified paramaters}
 		\end{itemize}
 		\item{Verify}
 		\begin{itemize}
 			\item{Compares the program output with the pre-defined expected result}
 			\item{A Unit Test is only as good as the accuracy of the pre-defined test result}
 		\end{itemize}
 		\item{Teardown}
 		\begin{itemize}
 			\item{Deletes data created during the test process like folders or code}
 		\end{itemize}
 	\end{itemize}
 	\subsubsection{JUnit}
 	JUnit is a framework to write repeatable tests for Java files. To write a test a specific test class has to be created. Before you can use JUnit you need to import it. Within Maven you need to import the three artifacts \texttt{junit-jupiter-api}, \texttt{junit-jupiter-engine} and \texttt{junit-jupiter-params} all coming from the \texttt{org.junit.jupiter} group. The class then requires three Decorators:
 	\begin{itemize}
 		\item{@BeforeEach (@Before in JUnit 4)}
 		\begin{itemize}
 			\item{Specify what should be done before the test is executed}
 			\item{This can include the creation of new classes or the definition of variables}
 		\end{itemize}
 		\item{@AfterEach (@After in JUnit 4)}
 		\begin{itemize}
 			\item{Specifiy what should be done after the tests have concluded}
 		\end{itemize}
 		\item{@Test}
 		\begin{itemize}
 			\item{Specifies the tests to be executed.}
 			\item{Within this test environment you should use \texttt{Assert.assertTrue()} or other assert functions to check if the output is as expected. Only failed asserts are recorded to immediately see tests that have failed.}
 		\end{itemize}
 	\end{itemize}
 	For small tests it isn't necessary to specifically define BeforeEach and AfterEach. Instead you can just call @Test, create the necessary variables and immediately run the function to be tested as an argument of the assert function.
\subsubsection{Conflicts}
	When a Three-Way-Merge isn't possible, Git prompts the user to resolve the conflict. To make it easier, Git sections the conflicting changes so you can see what is conflicting with what.
	\subsubsection{Workflows}
	Git expects various workflows to make collaboration and versioning easier. Some of them are:
	\paragraphlb{Feature Branch}
	Feature Branches should be created for every new feature to be added to the project. This means ideally that the main branch will never contain broken code, while all development is done in branches. These are after a pull request merged into main.
	\paragraphlb{Gitflow}
	An older workflow, the gitflow workflow has the main branch only contain the official release history. Meanwhile the develop branch is used for new fatures and integrates into the main branch (without merging) for every major version.
	\paragraphlb{Forking}
	Every maintainer maintains their own fork of a remote repository. Thus there is a private local repository and a public one. New features are then merged into the public repository as needed. This has the advantage that you don't need to maintain rights for every user working on a project.
	\subsection{Writing good commit messages}
	Since commit messages mainly serve to tell you and others what you did for this commit, it is important to structure them well. A good commit message communicates the context of changes. The 7 rules of good commit messages are:
	\begin{itemize}
		\item{Add a line between the summary and the commit main text}
		\begin{itemize}
			\item{Any commit message consisting of more than the heading should add one line of empty space.}
		\end{itemize}
		\item{The summary should be less than 50 characters}
		\begin{itemize}
			\item{This is a rough rule and doesn't have to be followed by the dot. If the summary needs to be this long then it might be a good idea to commit more often}
		\end{itemize}
		\item{The Summary should start with a capital letter}
		\item{The commit message does not need a full stop at the end}
		\item{Use the imperative mood in summary}
		\begin{itemize}
			\item{Commit messages should be worded like a command: \texttt{Merge branch "feature42"}}
			\item{This is because English always fronts a sentence with the verb if it is in the imperative: (Compare \texttt{Merge branch "feature42"} to \texttt{"Feature42" branch merged.})}
			\item{In general a summary should complete the following sentence: \texttt{If applied, this commit will ...}}
		\end{itemize}
		\item{The main text should have a line break after 72 characters}
		\item{In the main text describe what and why instead of how.}
	\end{itemize}
	\subsection{Code Comments}
	Comments in code can help significantly in understanding what was done even for the programmer after a certain amount of time. But code comments can even be detrimental. Comments should:
	\begin{itemize}
		\item{Not state the obvious}
		\item{Not excuse unclear code}
		\begin{itemize}
			\item{If possible it is usually better to rewrite code to make it more readable than add comments because it is not immediately obvious}
		\end{itemize}
		\item{Be clear and concise}
		\begin{itemize}
			\item{If that proves difficult, it might be necessary to refactor the code}
		\end{itemize}
		\item{Not cause more confusion}
		\begin{itemize}
			\item{Comments should dispell confusion instead of causing it}
		\end{itemize}
		\item{Explain unidiomatic features}
		\begin{itemize}
			\item{Occasionally code may look poorly written. If it is like that on purpose, add a comment to explain your reasoning}
		\end{itemize}
		\item{Link to the original source of code}
		\begin{itemize}
			\item{Linking to a copied solution allows the reader to get more context from the original source or update it in the future}
		\end{itemize}
		\item{Link to outside reference}
		\item{Mark incomplete implementation}
		\begin{itemize}
			\item{If something still needs to be done, mark it clearly with a comment. You might know what to do and where but will you in a year or anybody who wishes to implement it?}
		\end{itemize}
	\end{itemize}
	\section{Documentation}
	Documentation is the User Guide to a program. In a survey of Open Source Projects, incomplete or confusing documentation was the most often cited reason for not engaging in a project. \\
	But it is not just other users who require documentation; just like code comments, documentation serves to bring everyone, including yourself, up to speed on the application.
	\subsection{The Grand Unified Documentation Theory}
	Daniele Procida created the Grand Unified Documentation Theory using four different kinds of dimensions of documentation. There is no single documentation useful for everybody but you have to know your audience and adjust your documentation based on that. It splits the step on a two-dimensional scale. Whether documentation is meant for a practical use case or purely theoretical or used by someone who uses it to study or to work with it. Tutorials, for example, are useful when you need practical knowledge while also mainly focusing on studying the matter. On the other hand a How-To Guide is also focusing on the practical aspect but for somebody who needs it for work. The four archetypes are:
	\subsubsection{Tutorials}
	A tutorial is meant as lessons taken by the reader by hand to complete a small series of steps to achieve a limited goal. In this relationship the writer of the tutorial takes the role of a teacher. It is very oriented on learning how to do something instead of learning theoretical knowledge. An example of a tutorial is a children's cook book as it starts from 0 to engage the child. \\
	Tutorials should:
	\begin{itemize}
		\item{Allow the user to learn by doing}
		\item{Get the user started with minimal info}
		\item{Make sure it works and is up-to-datte}
	\end{itemize}
	\subsubsection{How-To Guide}
	Focus on taking the reader through steps to solve a real world problem. This means that the guide can already assume some knowledge on the user's end. These guides are very goal-oriented. A cooking example would be a recipe. It assumes you have the knowledge you do it and just shows you the specific steps. \\
	How-To Guides should include:
	\begin{itemize}
		\item{Provide a series of steps}
		\item{Focus on results}
		\item{Meant to solve a particular problem}
		\item{Don't explain concepts but how to use them}
		\item{Allow flexibility}
	\end{itemize}
	\subsubsection{Reference}
	Reference is on the theoretical side and offers technical descriptions and how to operate something. It is very information oriented, offering a complete as possible guide to do something. This is often seen as the 'only' kind of documentation. \\
	As a cooking example this would be a taste encyclopaedia. It doesn't show any specific knowledge but only how to generally use something. \\
	References should include:
	\begin{itemize}
		\item{Structure the documentation around the code described}
		\item{Be consistent}
		\item{Be accurate and complete}
		\item{Only describe}
	\end{itemize}
	\subsubsection{Explanation}
	The likely least explicitly created documentation, explanations give background information on a topic. In this way they are often created together with other documentation. A cooking example would be a context book, talking about how practices have evolved over time in different cultures. \\
	Explanations should include:
	\begin{itemize}
		\item{Provide context}
		\item{Docuss alternatives and opinions}
		\item{No instructions or technical reference}
	\end{itemize}
	\subsubsection{JavaDoc}
	Java supports three kinds of comments:
	\begin{itemize}
		\item{\texttt{// single line comment}}
		\item{\texttt{/* multiline comment */}}
		\item{\texttt{/** JavaDoc comment */}}
	\end{itemize}
	JavaDoc allows the automatic creation of Java Documentation. By adding a comment before the method referenced, it can immediately convert it to finished documentation. From this comment it also creates the documentation found for the Java Docs. There's a few specific parameters that can be added if needed:
	\begin{itemize}
		\item{@param}
		\begin{itemize}
			\item{The parameters passed to the method}
		\end{itemize}
		\item{@return}
		\begin{itemize}
			\item{The values a method can return}
		\end{itemize}
		\item{@throws}
		\begin{itemize}
			\item{The possible exceptions the method can throw.}
		\end{itemize}
	\end{itemize}

























	
\end{document}