\documentclass{article}

\usepackage{geometry}
\usepackage{makecell}
\usepackage{array}
\usepackage{multicol}
\usepackage[ngerman]{babel}
\usepackage{setspace}
\usepackage{changepage}
\usepackage{booktabs}
\usepackage[explicit]{titlesec}
\usepackage{hyperref}
\usepackage{graphicx}
\usepackage{cprotect}
\usepackage{float}
\newcolumntype{?}{!{\vrule width 1pt}}
\newcommand{\paragraphlb}[1]{\paragraph{#1}\mbox{}\\}
\newcommand{\subparagraphlb}[1]{\subparagraph{#1}\mbox{}\\}
\renewcommand\theadalign{tl}
\setstretch{1.10}
\setlength{\parindent}{0pt}

\geometry{top=12mm, left=1cm, right=2cm}
\title{\vspace{-1cm}Datenstrukturen und Algorithmen: Übung 1 - Änderungen}
\author{Andreas Hofer}

\begin{document}
	\maketitle
	\section*{Änderungen}
	\subsection*{Entfernen eines Lieds in der Mitte der Playlist}
	In meinen Tests hatte ich nie mehr als 2 nodes, weshalb das zu entfernende Element nie weder das Erste noch das Letzte war. Jetzt sollte es wie erwartet funktionien.
	\subsection*{Dead Code}
	Ich habe first, last und setAdjacent im Vorhinein implementiert und dann übersehen, dass sie gar keinen Nutzen haben. Sie wurden entfernt
	\subsection*{numberSpaces in toString}
	Gibt jetzt eine beliebige Menge an Leerzeichen zurück, da die Menge mittels Logarithmus berechnet wird.
	\subsection*{enum für longestString}
	Das Argument für longestString ist jetzt ein enum und heißt entweder Path oder Name
	\subsection*{Kommentare in toString}
	Ich habe die Formattierung in toString geändert und ein paar Kommentare hinzugefügt um es eventuell lesbarer zu machen


	























  
\end{document}