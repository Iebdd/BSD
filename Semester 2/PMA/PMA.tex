\documentclass{article}

\usepackage{geometry}
\usepackage{makecell}
\usepackage{array}
\usepackage{multicol}
\usepackage{setspace}
\usepackage{nicefrac}
\usepackage{changepage}
\usepackage{booktabs}
\usepackage[explicit]{titlesec}
\usepackage{hyperref}
\usepackage{graphicx}
\usepackage{cprotect}
\usepackage{float}
\newcolumntype{?}{!{\vrule width 1pt}}
\newcommand{\paragraphlb}[1]{\paragraph{#1}\mbox{}\\}
\newcommand{\subparagraphlb}[1]{\subparagraph{#1}\mbox{}\\}
\renewcommand{\contentsname}{Inhaltsverzeichnis:}
\renewcommand\theadalign{tl}
\setstretch{1.10}
\setlength{\parindent}{0pt}

\titleformat{\section}
  {\normalfont\Large\bfseries}{\thesection}{1em}{\hyperlink{sec-\thesection}{#1}
\addtocontents{toc}{\protect\hypertarget{sec-\thesection}{}}}
\titleformat{name=\section,numberless}
  {\normalfont\Large\bfseries}{}{0pt}{#1}

\titleformat{\subsection}
  {\normalfont\large\bfseries}{\thesubsection}{1em}{\hyperlink{subsec-\thesubsection}{#1}
\addtocontents{toc}{\protect\hypertarget{subsec-\thesubsection}{}}}
\titleformat{name=\subsection,numberless}
  {\normalfont\large\bfseries}{\thesubsection}{0pt}{#1}

\hypersetup{
    colorlinks,
    citecolor=black,
    filecolor=black,
    linkcolor=black,
    urlcolor=black
}

\geometry{top=12mm, left=1cm, right=2cm}
\title{\vspace{-1cm}Personalmanagement und Arbeitsrecht}
\author{Andreas Hofer}

\begin{document}
	\maketitle
	\tableofcontents
	\section{Personalmanagement}
	Personalmanagement stellt den ganzheitlichen Ansatz des Managements von Mitarbeitern dar. Sie verknüpft die Unternehmensstrategie mit den Anforderungen des Personals. Dabei gibt es einen Unterschied zwischen Personalverwaltung und Personalmanagement. Während die Personalverwaltung rein auf die Verwaltung des Personals zielte und so die Effizienz der Arbeitenden als Ziel hatte, wurde dies mit der Zeit durch das Personalmanagement ersetzt, welches nicht nur auf die Wirtschaftlichkeit des Unternehmens sondern auch auf die Zufriedenheit der Mitarbeiter zielt. Da ein starker Wettbewerb um Talente geführt wird, muss man als attraktiver Arbeitgeber wirken. \\
	Das Personalmanagement hat sich mit der Zeit einem großen Wandel unterzogen. Vor Beginn des 20. Jahrhunderts war Personalverwaltung die vorherrschende Arbeitsweise, welche auf Effizienz und Produktivität zielte. Dabei wurden Menschen als austauschbare Produktionsfaktoren gesehen. Mit den 1930ern kamen Human Relation mehr in den Fokus und die Zufriedenheit des Arbeitnehmers wurde stärker gefördert. Ausschlaggebend war hier das Hawthorne-Experiment, welches zu dem Schluss kam, dass die Untersuchung der Zufriedenheitsfaktoren allein bereits zu einer Effizienzsteigerung führt. Mit den 1990ern wurde das Harvard/Michigan Modell vorherrschend in welchem HR als Wettbewerbsvorteil gesehen wurde. \\
	\subsection{Aufgaben des Personalmanagements}
	HR besteht aus einer Reihe an Prozessen, welche langfristig Wertschöpfung generieren sollen. Dabei werden diese in drei Teile klassifiziert: Die steuernden HR-Prozesse, die operativen HR-Prozesse sowie Service HR-Prozesse. Steuernde Prozesse bestehen aus Führung sowie Marketing des Personals. Operative Prozesse bestehen aus der Personalplanung, Beschaffung sowie Entwicklung (und auch Entlassung). Serviceprozesse bestehen aus weiterführenden Tätigkeiten wie Controlling. \\
	Das Säulenmodell von Dave Ulrich bildet in vielen Unternehmen die Grundlage der Personalabteilung: HR Business Partner, Center of Expertise sowie Shared Service Center. HR als Business Partner unterstützt Führungskräfte bei Talentmanagement und Nachfolgeplanung und der Entwicklung sowie Umsetzung von HR Strategien. Das Center of Expertise gilt zur Entiwcklung von Richtlinien, Programmen und Systemen sowie die Bereitstellung von Best Practices. Shared Services verwalten Routineprozesse und Transaktion und betreiben oft spezifische Servicezentren. Shared Services werden manchmal auch an andere Firmen ausgelarget um sich auf 'wichtigere' teile des HR zu fokusieren. \\
	Das deutsche Zukunftsinstitut veröffentlicht jährlich die Trends der Gesellschaft. Im Moment sind solche Trends unter anderem ein Gender Shift, mehr Fokus auf Gesundheit, eine erhöhte Globalisierung sowie Individualisierung. Unter anderem auch der Fokus darauf, dass die Gesellschaft immer älter wird oder mehr Personen in Städten leben. HR muss solche Trendänderungen als Faktoren miteinbeziehen und sich darauf vorbereiten. Zum Beispiel, dass spezifisch das Wissen älterer Mitarbeiter in der Firma erhalten wird, da sonst, sobald viele Leute gleichzeitig in Pension gehen, dieses Wissen eventuell verloren geht. \\
	Kernaufgaben des Personalmanagements bilden die Einstellung und Bindung von Personal und dessen strategischer Einsatz Entwicklung und Entlohnung.
	\subsection{Personalplanung}
	\begin{quote}
	 	"...having the right people with the right skills in the right place at the right time."
	 \end{quote} 
	 Die Personalplanung befasst sich mit der Verwaltung von Mitarbeitern und stellt sicher, dass die Anzahl der Mitarbeiter ausreicht, diese die richtigen Qualifikationen besitzen, diese zur rechten Zeit arbeiten und am richtigen Ort sind, wenn sie benötigt werden. Diese vier Dimensionen bilden den Grundsatz für die Personalplanung.
	 \subsubsection{Personalbedarf}
	 Es gibt viele mögliche Gründe für Personalbedarf. Mitarbeiter können unvorhergesehen kündigen, erkranken oder versterben. Zusätzlich gehen Mitarbeiter kontinuierlich in Pension oder Karenz. Oft gibt es auch Mehrbedarf um flexibel Personen einsetzen zu können. \\
	 An oberster Stelle bei der Planung besteht die Wirtschaftlichkeit. Das wird gestützt durch die Leistungssicherung um sicherzustellen für die Zukunft gewappnet zu sein. Man sollte jedoch aufpassen, dass die Arbeitsbelastung ausgeglichen und angemessen ist, da man immer verhindern will, dass ein Mitarbeiter wegen Burnout ausfällt (und ein anderer eventuell keine Arbeit hat). \\
	 Der Personalbedarf hat eine Vielzahl an Einflussfaktoren. Diese können den Bedarf erhöhen oder verringern. Einige externe Einflussfaktoren sind demografische Entwicklungen sowie konjunkturelle Einflüsse. Eventuell kann sich auch die Marktstruktur oder das Arbeitsrechts verändern. (Zum Beispiel erhöhte oder verringerte Regelarbeitszeiten). Diese Faktoren passieren und ein Unternehmen kann sich diesen meist nur anpassen. Interne Faktoren können eine Umgestaltung des Produktionsprogramms oder eine Verbesserung der Kommunikations- Fertigkeits- oder Informationstechnologien. Mit der Zeit können sich auch Mitarbeiterinteressen wandeln. Während der Arbeitgeber darauf Einfluss nehmen kann, können marktweite Änderungen diese auch nötig machen. \\
	 Der Personalbedarf ergibt sich aus einem Vergleich des Bestands und des Bedarfs. So kann es entweder zu einer Überdeckung oder einer Unterdeckung kommen und Mitarbeiter sollten entweder rekrutiert oder freigestellt werden. Es gibt viele Verfahren um diesen Bedarf zu berechnen, welche aus qualitativen oder quantitativen Faktoren beruhen kann. \\
	 \subsection{Personalbeschaffung}
	 Personalbeschaffung (oftmals nur mehr Recruiting genannt) befasst sich mit der Rekrutierung neuer Mitarbeiter. Dabei gibt es Faktoren, welche die Wege und Strategien bestimmen. Wie ist die Situation am Arbeitsmarkt? Was ist das gesuchte Qualifikationsprofil? Wie hoch ist das Budget? Wie dringlich ist der Bedarf? Wie lange benötigt man diese Leistung? Man muss zwischen diesen Faktoren den besten Weg wählen. \\
	 Es gibt auch viele Quellen um Personal zu beschaffen. Dabei kann man entweder intern oder extern rekrutieren. \\
	 Intern kann man mit oder ohne Personalbewegung rekrutieren. Ohne, wenn man bereits existierenden Mitarbeitern Mehrarbeit oder Überstunden auferlegt. Mit, wenn man innerbetrieblich versetzt oder befördert. Zusätzlich kann dies auch mit einem Nachfolgeplan geschehen, wobei man spezifisch für eine Position ausbildet.\\
	 Extern unterscheidet man zwischen passiver und aktiver Personalbeschaffung. Passiv kann man über eine Agentur oder einen Arbeitsmarktservice rekrutieren und immer Initativbewerbungen akzeptieren. Aktive Personalbeschaffung funktioniert über aktive Stellenanzeigen und Karrieremessen um Talente zu identifizieren. Institutionelle Kontakte können dabei auch relevant werden. \\
	 \subsubsection{Anforderungsprofil}
	 Bei der Beschaffung existieren stets Anforderungen, welche in Muss-, Kann- und Sollkriterien geteilt werden. Dabei muss sich die Personalabteilung mit den suchenden Mitarbeitern abstimmen um sicherzustellen, dass auch wirklich die richtige Person mit den richtigen Skills rekrutiert wird.
	 \subsubsection{Recruiting}
	 Personalbeschaffung ist bereits ein relativ veralteter Begriff. Heutzutage wird oft stattdessen von Recruiting gesprochen, da dadurch der Fokus weg von Mitarbeitern als Ressource geht. Der Prozess sieht jedoch relativ ähnlich aus.
	 Der Prozess des Recruitings hat sich mit der Zeit gewandelt. Während davor Recruitingunternehmen und Zeitungswerbung vorherrschend waren, sind heutzutage Onlineprofile von Unternehmen sowie Onlinejobportale das wichtigste Medium. Man kann klar den Fokus auf das Internet auch für Recruiting sehen.
	 \subsubsection{Key Performance Indicators (KPI)}
	 Auch bei dem Recruiting gibt es KPIs um zu sehen wie effizient Personen eingestellt werden. Dabei unterscheidet man zwischen Time-to-Fill, welche von der Anmeldung des Bedarfs bis zur Einstellung geht und Time-to-Hire welche von der Aussendung der Recruitingunterlagen bis zum Stellenangebot geht. Es gibt noch weitere Indikatoren, welche von Zeit über Kosten bis Qualität geht. Zusätzlich wird auch die Effektivität von Recruitingkanälen untersucht. \\
	 \subsubsection{Trends}
	 Auch bei Recruiting gibt es Trends. Dabei tritt künstliche Intelligenz wie so oft in den Fokus. Ebenfalls wird mehr Wert auf die Fähigkeiten der Mitarbeiter anstatt der formellen Ausbildung gelegt. Remote Work wird auch bedeutend
	 \subsubsection{Employer Branding}
	 Das Employer Branding beschreibt Maßnahmen um die eigene Arbeitgebermarke zu stärken und sich so am Arbeitsmarkt besser aufzustellen. Dabei gibt es internes und externes Employer Branding. Wenn es an Mitarbeiter gerichtet ist, stärkt es die Bindung des Mitarbeiters an das Unternehmen sowie deren Zufriedenheit. Es kann auch helfen Mitarbeiter zu behalten, und sich fürtführend als besten Arbeitgeber zu positionieren. Externes Branding versucht potenzielle Mitarbeiter für die Firma zu interessieren und ein positives Image aufzubauen. Diese beiden Faktoren arbeiten eng zusammen, da man die nach außen projezierte Marke auch intern fortführen muss um das angeworbene Talent behalten zu können. \\
	 Das Branding durchläuft einen Plan um potenzielle Mitarbeiter nicht nur zu interessieren sondern auch aufzunehmen und zu behalten. Selbst wenn der Mitarbeiter das Unternehmen wieder verlässt, muss das nicht bedeuten, dass es wegen schlechter Erfahrungen geschehen ist.
	 \subsection{Personalauswahl}
	 Wenn das Branding on-point ist, erhält man viele Bewerbungen und muss passende Personen auswählen. Dabei spielen einige Kriterien eine Rolle: Hat der Bewerber die nötigen Skills? Hat der Bewerber auch nice-to-have Skills? Welche Erfahrung kann der Bewerber mitbringen? Passt der Berwerber in die Unternehmenskultur? Hat der Bewerber einen relevanten Studienabschluss?
	 \subsubsection{Professional vs Cultural Fit}
	 Man unterscheidet hierbei zwischen dem Professional und dem Culture Fit. Man 'passt' auf einer professionellen Ebene wenn das Gehalt sowiedie karrieretechnischen Vorstellungen zusammenpassen. Ebenso wird dabei ein Wert auf Erfahrungen und Fertigkeiten gelegt. Dem gegenüber steht es kulturell zu passen, was davon abhängt, wie das Klima in der Firma ist und wie die Person dort hineinpasst. Sind die Werte der Firma die selben wie die des Bewerbers? Was sind die individuellen Werte? \\
	 Diese beiden fits besitzen annähernd ähnliche Wichtigkeit, da jemand der alle Vorraussetzungen erfüllt, aber nicht mit den anderen Mitarbeitern zusammenpasst auch kein guter Mitarbeiter ist.
	 \subsubsection{Faktoren}
	 Bei der Auswahl wird zuerst vorab entschieden und danach die Endauswahl getroffen. Voraus werden Empfehlungen von Mitarbeitern in Betracht gezogen und Bewerbungsunterlagen sowie weitere Instrumente wie Fragebogen oder Interviews überprüft. Die Endauswahl geschieht letztendlich nach einem Bewerbungssgespräch sowie Testeverfahren. Diese können Leistungstests oder psychologische Tests sein. Bei technischen Berufen beinhaltet das auch das technical Interview. Teil der Endauswahl ist auch eine Arbeitsprobe sowie arbeiten auf Probezeit.
	 \subsubsection{Integration}
	 Nachdem man eine Person eingestellt hat, muss man diese onboarden und in die Firma integrieren. 30\% aller neuen Mitarbeiter verlassen das Unternehmen wieder weil der Onboardingprozess unzureichend war. Dabei teilt man das in drei Teile:
	 \begin{itemize}
	 	\item{Der Preboardingphase}
	 	\begin{itemize}
	 		\item{Sachen die eigentlich schon vor der Anstellung geschehen sein sollen.}
	 		\item{Beinhaltet equipment das benötigt wird oder ein Buddy der zur Orientierung zugeteilt wird.}
	 	\end{itemize}
	 	\item{Der Orientierungsphase}
	 	\begin{itemize}
	 		\item{Ein neuer Mitarbeiter wird in das Unternehmen eingebracht und erhält einen Paten für Fragen. So hat er eine Ansprechsperson um eventuell anfallende Probleme lösen zu können.}
	 	\end{itemize}
	 	\item{Der Integrationsphase}
	 	\begin{itemize}
	 		\item{Geschieht nach etwa einem bis drei Monaten.}
	 		\item{Interview um die Einschätzung des Mitarbeiters einzuholen und herauszufinden ob es Probleme gibt.}
	 	\end{itemize}
	 \end{itemize}
	 \subsection{Personalentwicklung}
	 Die Personalentwicklung umfasst die Entwicklung des Personals um deren Potential zu erweitern. Dabei geht es um Weiterbildungen, welche von dem Unternehmen angeboten werden. Relevant für diese Strategie sind natürlich die Unternehmens- sowie Mitarbeiterziele. Fortbildungen sollten deshalb in die Strategie des Unternehmens passen, jedoch auch Mitarbeiter mit Potential gezielt ansprechen. Viele Unternehmen bieten deshalb eigene 'Academies' an, welche auf die Anforderungen des Unternehmens zugeschnitten sind. \\
	 Die PE verfolgt Ziele, weleche in drei Kategorien geteilt werden können:
	 \begin{itemize}
	 	\item{Unternehmerperspektive}
	 	\begin{itemize}
	 		\item{Dient zur Sicherung des Fachbestands der Firma}
	 		\item{Erkennen und vorbereiten von Führungskräften und Spezialisten}
	 		\item{Bessere Unabhängigkeit von externen Arbeitsmärkten}
	 		\item{Bessere Motivation}
	 	\end{itemize}
	 	\item{Mitarbeiterperspektive}
	 	\begin{itemize}
	 		\item{Verbesserung der persönlichen Qualifikation}
	 		\item{Verbesserung der Position am Arbeitsmarkt}
	 		\item{Bessere Karriere und Laufbahnmöglichkeiten}
	 		\item{Verbesserung des Einkommens}
	 	\end{itemize}
	 	\item{Gesellschaftspolitische Perspektive}
	 	\begin{itemize}
	 		\item{Langfristige Beschäftigungssicherung}
	 		\item{Bessere Allokation von der Fähigkeiten der Bevölkerung}
	 		\item{Persönlichkeitsentfaltung}
	 		\item{Chancengleichheit und soziale Mobilität}
	 	\end{itemize}
	 \end{itemize}
	 \subsubsection{Maßnahmen}
	 Es gibt unterschiedliche Maßnahmen um diese Fortbildung auszuführen:
	 \begin{itemize}
	 	\item{On the Job}
	 	\begin{itemize}
	 		\item{Weiterbildung innerhalb des Unternehmens während des Arbeitens mittels Projekten}
	 	\end{itemize}
	 	\item{Along the Job}
	 	\begin{itemize}
	 		\item{Einsatz von Stellvertretung oder Assistenz. Dient zur Karriereplanung oder einem Nachfolgeplan}
	 	\end{itemize}
	 	\item{Out of the Job}
	 	\begin{itemize}
	 		\item{Zum Ersatz von Mitarbeiten, welche in den Ruhestand gehen.}
	 	\end{itemize}
	 	\item{Off the Job}
	 	\begin{itemize}
	 		\item{'Klassische' Fortbildungsmöglichkeiten wie Fortbildungen oder ein Studium}
	 	\end{itemize}
	 	\item{Near the Job}
	 	\begin{itemize}
	 		\item{Verbindet on und off the job indem zwar abseits Wissen vermittelt wird, dies aber direkten Bezug auf die Position nimmt.}
	 	\end{itemize}
	 	\item{Into the Job}
	 	\begin{itemize}
	 		\item{Einarbeitung in eine Sparte wie Traineeprogramme oder Praktika}
	 	\end{itemize}
	 \end{itemize}
	 \subsubsection{New Learning}
	 Die Weise der Fortbildung hat sich in letzter Zeit grundlegend gewandelt. Während früher formelle Institutionen wie Universitäten die Lehre exklusiv übernahmen, gibt es heute viele weitere formelle und informelle Möglichkeiten sich weiterzubilden. Während früher Lehrbücher und Materialien von Lehrenden die einzige Informationsquelle waren, ist es heute möglich im Internet nahezu alles näher kennenzulernen bzw. dieses kritisch zu hinterfragen. Während früher Lehrende explizit Wissen weitergaben, wird heute auch ein passendes Umfeld vorrausgesetzt. Während früher einheitliche Lehrpläne existierten, können Lehrmethoden heute individuell angepasst werden.
	 \subsubsection{Kompetenzorientierung}
	 Kompetenz ist die Fähigkeit Wissen und Können so zu verbinden, dass Anforderungen am Arbeitsplatz selbstständig und eigenverantwortlich bearbeitet werden können. Dabei unterscheidet man zwischen der Fachkompetenz, der Selbstkompetenz, der Sozialkompetenz und der Methodenkompetenz. \\
	 Die Fachkompetenz bezeichnet Fachspezifische Kompetenzen wie Sprachen, Kenntnisse von fachspezifischen Standards und Regeln sowie technisches Know-How. Selbstkompetenz beschreibt die Belastbarkeit, Eigenverantwortung und Lernbereitschaft. Sozialkompetenz ist Emptahie, Menschenkenntnis sowie Konfliktlösung. Und Methodenkompetenz ist die Fähigkeit Wissen einzuholen, zu strukturieren und weiterzugeben zum Beispiel in einem Projektmanagementkontext. \\
	 Um Kompetenzen zu kategorisieren wurde der Kode Kompetenzatlas entwickelt, welcher Metakompetenzen abbildet und diese auf 64 detaillierte Schlüsselkompetenzen beschreibt. \\
	 Jedes Unternehmen erstellt so anhand der Unternehmensstrategie einen eigenen Kompetenzatlas ab. Anhand von diesem werden Aus- und Fortbildung angepasst und schließlich evaluiert, ob es zu einem besseren Ergebnis geführt hat. Bei einer Kompetenzorientierung kann Personalentwicklung gezielter geschehen sowie neue Mitarbeiter gezielter ausgewählt werden.
	 \subsection{Personalführung}
	 Personalführung ist der zwischenmenschliche Prozess, in welchem Führungskräft Mitarbeiter zielführend beeinflussen. Führungskräfte sind auch für Unterstützung, Förderung und Motivation zuständig. Dabei gibt es nach Malik(2000) 5 Aufgaben wirksamer Fürhung:
	 \begin{itemize}
	 	\item{Organisieren}
	 	\item{Entscheiden}
	 	\item{Kontrollieren}
	 	\item{Entwickeln und Fördern}
	 	\item{Konflikte lösen}
	 \end{itemize}
	 Daran sieht man, dass Führungskräfte nicht mehr nur Mitarbeiter kontrollieren und überwachen sollen, sondern dies auch aufbauen. Das baut auf der Theorie der Menschenbilder. Diese ging vom 'Economic Man', dem rein finanziell getriebenen Mitarbeiter zum 'Digital Man', einem Mitarbeiter, welcher bei der Arbeit Selbstverwirklichung sucht. \\
	 Douglas McGregor definierte die Theorie X und Y um diesen abzubilden. Theorie X nimmt an, dass ein Mitarbeiter stets kontrolliert werden muss damit er produktiv ist und permanent genaue Anweisungen benötigt. Theorie Y hingegen nimmt an, dass ein Mitarbeiter intrinsisch motiviert ist und produktiv sein will, weshalb man diese individuell aufbauen und ihnen Freiheiten lassen muss damit sie sich vol lentfalten können. So ging man von einem komplett autoritäten zu einem partizipativen Modell über. \\
	 Eine andere Theorie existiert von Edgar Schein, welcher den Mensch in vier Kategorien einteilt, wobei jede Art Mensch eine andere Art von Führungsstil benötigt:
	 \begin{itemize}
	 	\item{Der rational-ökonomische Mensch}
	 	\begin{itemize}
	 		\item{Wird durch monetäre Anreize motiviert und ist sehr passiv}
	 		\item{Die Organisation manipuliert, motiviert und kontrolliert}
	 		\item{Benötigt klassische Managementfunktionen und arbeitet nicht unabhängig}
	 	\end{itemize}
	 	\item{Der soziale Mensch}
	 	\begin{itemize}
	 		\item{Wird durch soziale Bedürfnisse motiviert}
	 		\item{Gruppenzugehörigkeit hat hohen Stellenwert}
	 		\item{Benötigt soziale Anerkennung durch die Gruppe und den Manager. Interessiert an Aufbau und Förderung der Gruppe}
	 	\end{itemize}
	 	\item{Der sich selbst verwirklichende Mensch}
	 	\begin{itemize}
	 		\item{Strebt nach Selbstverwirklichung und bevorzugt Selbstmotivation}
	 		\item{Führung sollte sich an die Interessen des Menschen anpassen, welche Unterstützer und Förderer sind}
	 	\end{itemize}
	 	\item{Der komplexe Mensch}
	 	\begin{itemize}
	 		\item{Ist lernfähig und flexibel und handelt situationsbezogen}
	 		\item{Führungkraft soll Situation diagnostizieren und das Verhalten situativ anpassen}
	 	\end{itemize}
	 \end{itemize}
	 Aus dieser Theorie entstand das Führungskontinuum nach Tannenbaum/Schmidt aus 1958, ist also schon lange in der Theorie. So sind Führungsstile keine abgetrennten Schubladen mehr, sondern existieren in einer laufenden Skala:
	 \begin{itemize}
	 	\item{Autoritärer Führungsstil}
	 	\begin{itemize}
	 		\item{Die Führungskraft entscheidet und ordnet an}
	 	\end{itemize}
	 	\item{Patriarchalisch}
	 	\begin{itemize}
	 		\item{Die Führungskraft entscheidet, ist aber bestrebt Mitarbeiter zu überzeugen}
	 	\end{itemize}
	 	\item{Beratend}
	 	\begin{itemize}
	 		\item{Führungskraft entscheidet, erbittet aber Fragen und Stellungnahmen}
	 	\end{itemize}
	 	\item{Konsultativ}
	 	\begin{itemize}
	 		\item{Führungskraft entscheidet, Mitarbeiter werden einbezogen und rechtzeitig informiert und erbeten Vorschläge einzubringen}
	 	\end{itemize}
	 	\item{Partizipativ}
	 	\begin{itemize}
	 		\item{Die Führungskraft entscheidet sich für eine Möglichkeit aus Vorschlägen der Mitarbeiter}
	 	\end{itemize}
	 	\item{Delegativ}
	 	\begin{itemize}
	 		\item{Mitarbeiter entscheiden; Führungskraft koordiniert}
	 	\end{itemize}
	 	\item{Kooperativ}
	 	\begin{itemize}
	 		\item{Jedem Mitarbeiter wird individuelle Freiheit gegeben}
	 	\end{itemize}
	 \end{itemize}
	 Ein moderner Führungsstil ist so relativ unterschiedlich zu einem 'traditionellen' Führungsstil. Traditionell ergibt sich Macht durch die Position und übt so Autorität aus, modern existiert der Glaube aus Stärke durch Kollaboration. Traditionell wird Wissen und Information nur bei Bedarf ausgegeben, während es modern offen mit den Mitarbeitern geteilt wird. Traditionell wird Feedback nur ein Mal jährlich ausgegeben, während modern Mitarbeiter kontinuierlich gecoacht werden. Traditionell hat ein Mitarbeiter spezifische statische Rollen während modern Rollen und Verwantwortung flexibel anhand der Verantwortlichkeiten sind.
	 \subsubsection{Führungsgrundsätze}
	 Führungsgrundsätze sind generelle Verhaltensempfehlungen für das Zusammenleben und -arbeiten von Mitarbeitern. Diese Grundsätze sollen eine einheitliche Grundlage für das unternehmensweit gewünschte Führungsverhalten schaffen. Wenn man zum Beispiel einen relativ liberalen Fürhungsstil hat, sollte man keine autoritäre Führungskraft einstellen.
	 \subsubsection{Generationengerechte Führung}
	 Führungsstile können auch stark von dem Alter der Mitarbeiter abhängen. Wenn man ein rein junges Team hat, hat man wahrscheinlich einen anderen Führungsstil als mit Babyboomern. So wuchsen Babyboomer in einer Nachkriegswelt auf und das Streben nach materieller Sicherheit hatte den höchsten Stellenwert. Gen Z hat stattdessen einen bedeutend höheren Fokus auf Work-Life Balance und einer Arbeit mit Sinn.
	 \subsubsection{Motivationstheorien}
	 Motivation wird in der Psychologie als psychologischer Vertrag gesehen. Motivation wird durch Bedürfnisse und Möglichkeiten zu deren Befriedigung gesteuert. Maslov's Bedürfnispyramide beschreibt die Stufen der Bedürfnisse, welche auch an einem Arbeitsplatz angewandt werden können. Hierbei muss stets die Stufe darunter erfüllt sein, bevor man sich der nächsten widmen kann. Zum Beispiel ist die Arbeitskultur von geringem Interesse, wenn die Arbeit nicht gerecht entlohnt wird.
	 \begin{enumerate}
	 	\item{Psychologische Bedürfnisse}
	 	\begin{itemize}
	 		\item{Ein ergonomischer Arbeitsplatz und gesunde Ernährung}
	 	\end{itemize}
	 	\item{Sicherheitsbedürfnisse}
	 	\begin{itemize}
	 		\item{Eine faire Entlohnung sowie unbefristete Arbeitsverträge}
	 	\end{itemize}
	 	\item{Soziale Bedürfnisse}
	 	\begin{itemize}
	 		\item{Eine angenehme Arbeitskultur sowie Teambuilding}
	 	\end{itemize}
	 	\item{Individualbedürfnisse}
	 	\begin{itemize}
	 		\item{Lob, Anerkennung sowie Auszeichnungen bei der Arbeit}
	 	\end{itemize}
	 	\item{Selbstverwirklichung}
	 	\begin{itemize}
	 		\item{Weiterbildungsangebot und Aufstiegsmöglichkeiten}
	 	\end{itemize}
	 \end{enumerate}
	\subsection{Personalentlohnung}
	Die Personalentlohnung ist die Gestaltung aller materieller Anreize, welche offiziell als Ausgleich gewährt wird. Also ist die Entlohnung die Bezahlung für die Arbeit. 
	\subsubsection{Entgeltgerechtigkeit}
	Gerechtigkeit in der Entlohnung basiert auf verschiedenen Faktoren. So sollte man erwarten, dass man für den gleichen Input stets den gleichen Output erhält:
	\begin{itemize}
		\item{Anforderungsabhängig}
		\begin{itemize}
			\item{Gleicher Lohnfür gleiche Arbeit}
		\end{itemize}
		\item{Leistungsabhängig}
		\begin{itemize}
			\item{Gleicher Lohn für gleiche Leistung}
		\end{itemize}
		\item{Erfolgsabhängig}
		\begin{itemize}
			\item{Gleicher Lohn für gleichen Erfolg}
		\end{itemize}
		\item{Qualifikationsabhängig}
		\begin{itemize}
			\item{Gleicher Lohn für gleiche Qualifikation}
		\end{itemize}
		\item{Statusabhängig}
		\begin{itemize}
			\item{Gleicher Lohn für gleiche Bedürfnisse}
		\end{itemize}
	\end{itemize}
	\subsubsection{Vergütungsmanagement}
	Zusätzlich zu den materiellien Anreizen (Lohn), Aktienoptionen, gibt es immaterielle Anreize wie der Arbeitsinhalt oder das Arbeitsklima gesehen. Da die Entlohnung so vielseitig sein kann, gibt es das Vergütungsmangement, welches es sich zur Aufgabe macht die Vergütung bzw. die Vergütungsstrukturen zu Planen, Abzuschätzen und zu implementieren.
	\subsubsection{Total Rewards Framework}
	Das Total Rewards Framework beinhaltet so alle Vergütungen der Firma, wie die Work-Life Balance, Rabatte, Firmenwohnungen, Urlaub, Boni, Aktien, jährliche Boni oder anders beinhalten.
	\subsection{Personalfreisetzung}
	Die Personalfreisetzung befasst sich mit der Beendigung eines Arbeitsverhältnisses. Dabei kann diese Freisetzung betriebsbedingt oder mitarbeiterbedingt geschehen. Betriebsbedingt wird wiederum in vorhersehbare oder unvorhersehbare Umstände geteilt. Vorhersehbare Umstände sind eine Verlagerung der Produktion, Absatzschwankungen aufgrund der Saison oder Veränderung im Produkt- und Leistungsprogramm. Unvorhersehbare Umstände können konjunkturell bedingte Absatzeinbrüche sein, können jedoch auch durch Management- und Planungsfehler versursacht werden. \\
	Andererseits gibt es auch mitarbeiterbedingte Freisetzungen. Wenn zum Beispiel ein Mitarbeiter eine nachlassende Arbeitsleistung erbringt oder steigende Arbeitsanforderungen nicht erfüllen kann. 
	\subsubsection{Maßnahmen}
	Maßnahmen zur Freisetzung kann durch Änderung der Arbeitsverhältnisse, Verzicht auf Neueinstellungen oder durch Beendigung bestehender Arbeitsverhältnisse. So kann man Kosten einsparen um Mitarbeiter durch Teilzeit oder vertikale bzw. horizontale Umstrukturierung zu versetzen.
	\subsection{Ausblick}
	Personalmanagement ist im Moment im Wandel. Da man digital von überall arbeiten kann, muss man Remote Work miteinbeziehen. Digitalisierung wird auch stets wichtiger und die Onlinepräsenz
	\section{Arbeitsrecht}
	In der Regel gilt der Grundsatz, dass das speziellere Recht dem generelleren Recht vorsteht. Die generellste Rechtssprechung ist das Allgemeine Bürgerliche Gesetzbuch (ABGB) und wurde 1812 in Österreich erlassen. Dieses beschreibt das Allgemeine Privatrecht oder das "Bürgerliche Recht".
	Das Bürgerliche Recht befasst sich im allgemeinen mit den Rechten und Pflichten von Privatpersonen.
	\begin{itemize}
		\item{Allgemeiner Teil}
		\item{Schuldrecht}
		\item{Sachenrecht}
		\item{Familienrecht}
		\item{Erbrecht}
	\end{itemize}
	Daneben existieren die spezielleren Rechtssprechungen:
	\begin{itemize}
		\item{Unternehmensrecht}
		\item{Wertpapierrecht}
		\item{Arbeitsrecht}
	\end{itemize}
	Diese Vorlesung fokussiert sich auf Arbeitsrecht. Das Arbeitsrecht beschreibt das Verhältnis zwischen dem Arbeitgeber und dem Arbeitnehmer. Da die Machtverteilung in der Regel mehr auf der Seite des Arbeitgebers ist, versucht das Arbeitsrecht für dieses Ungleichgewicht zu kompensieren. Dies geschieht in der Regel mit Rechten der Arbeitnehmer und, speziell in Österreich, die Verhandlungen der Kollektivverträge. \\
	\subsection{Aufbau der Arbeitsrechtsordnung}
	Das Recht in der EU baut auf einem Stufensystem auf, wobei eine höhere Stufe über eine niedrigere Präzedenz nimmt. Auf höchster Stufe steht das EU-Recht, welches von den Mitgliedsländern in nationales Recht überbracht werden muss und auf niedrigster liegt das Landesrecht. \\
	\subsubsection{Günstigkeitsprinzip}
	Auf dem gleichen Prinzip basiert das Günstigkeitsprinzip. Anders als das ABGB gibt es nicht nur ein Arbeitsrecht und stattdessen basiert es auf fünf Stufen: Das Gesetz, die Verordnung, der Kollektivvertrag, die Betriebsvereinbarung und der Arbeitsvertrag. Dabei gilt stets das günstigere dieser Stufen, also das was für den Arbeiter besser ist. Zum Beispiel schreibt das Gesetz vor, dass die maximale Arbeitswoche 40 Stunden beträgt, Kollektivverträge jedoch in der Regel nur 38,5 zulassen. Aus diesem Grund greift das Maximum aus dem Kollektivvertrag. Man könnte auch einen Arbeitsvertrag über 16 Stunden unterzeichnen und dann würde dieser den Vorzug erhalten.
	\paragraphlb{Ausnahmen}
	Es gibt jedoch auch Ausnahmen für das Günstigkeitsprinzip: Das Ordnungsprinzip trifft zu, wenn Absolut zwingende Normen eingehalten werden müssen. 
	\subsection{Kollektivverträge}
	Kollektivverträge (KV) sind schriftliche Vereinbarungen zwischen Interessensvertreter der Arbeitsgeber und Arbeitnehmer. In Österreich existieren knapp 1000 Verträge. Der Kollektivvertrag regelt dabei:
	\begin{itemize}
		\item{Die Arbeitszeit}
		\item{Kündigungsfristen und -termine}
		\item{Lohn- bzw. Gehaltsordnung sowie Mindestlöhne}
		\item{Zulagen und Zuschläge}
		\item{Aufwandsentschädigungen}
		\item{Sonderzahlungen}
	\end{itemize}
	Obwohl es so viele verschiedene Kollektivverträge gibt, kommt in der Regel die Zugehörigkeit des Arbeitsgebers zur Geltung. Also gilt der Kollektivvertrag für die Sparte in der die Firma seinen Hauptumsatz macht. Es gibt jedoch einen Weg wie man Kollektivverträge trennen kann, nämlich indem man zwei Sparten einer Firma organisatorisch trennt. \\
	Diese große Menge an Kollektivverträgen führt dazu, dass es in Österreich keinen gesetzlichen Mindestlohn gibt. Das bedeutet jedoch, dass manche Branchen bedeutend weniger verdienen als sie sollten. Der Kollektivvertrag der Zeitungszusteller garantiert zum Beispiel nur 910€ im Monat bei einer Vollzeitbeschäftigung, was nahe des Existenzminimums ist.
	\subsection{Betriebsvereinbarung}
	Ab einer bestimmten Größe können Mitarbeiter einer Firma sich aufstellen lassen um einen Betriebsrat zu gründen. So kann man bereits ab 5 Mitarbeitern ein Betriebsrat gegründet werden. \\
	Betriebsvereinbarungen sind schriftliche Vereinbarungen zwischen dem Arbeitgeber und dem Betriebsrat. Dabei gibt es Verpflichtende und Freiwillige Betriebsvereinbarungen. Verpflichtende müssen über den Betriebsrat geregelt werden. Dadurch kann der Betriebsrat nein sagen und es kann nicht geschehen. Freiwillige Vereinbarungen können hingegen pro Person und auch ohne Betriebsrat abgeschlossen werden.
	\subsection{Vertrag}
	Ein Vertrag ist ein Abkommen zwischen zwei Parteien und können auf drei Wegen geschehen: Schriftlich, Mündlich oder Stillschweigend. Schriftliche Verträge verlangen einen Vertrag, welcher unterzeichnet werden muss. Mündliche können mit verbaler Zustimmung abgeschlossen werden. Stillschweigende Verträge können durch Gesten oder Kopfnicken abgeschlossen werden. \\
	Damit ein Vertrag abgeschlossen werden kann, muss es zuerst ein Anbot oder eine Offerte geben. Dabei wird ein Vorschlag unterbreitet, welcher danach angenommen werden muss. \\
	Ein Dienstvertrag kann ebenfalls schriftlich, mündlich und stillschweigend abgeschlossen werden. Einzige Ausnahme bieten Lehrverträge, welche stets schriftlich sein müssen. Wenn ein Vertrag jedoch mündlich oder stillschweigend abgeschlossen wurde, hat man Anrecht auf einen Nachweis in Form eines Dienstzettels. Ein Dienstzettel und ein Dienstvertrag können in der Regel den gleichen Inhalt haben, ein Dienstvertrag muss jedoch unterschrieben werden, während ein Dienstzettel mit einer Unterschrift nur zur Kenntnis genommen wurde. Da der Vertrag das unterste Glied des Arbeitsrechts bildet, muss er ebenso das Günstigkeitsprinzip erfüllen. Sollte ein Teil des Vertrags dieses verletzen, wird dieser Teil ungültig, während der Rest des Vertrags bestehen bleibt. Diesen Vorgang nennt man Teilnichtigkeit, welcher nur im Arbeitsrecht als solches gilt. Gleichzeitig darf man in einem Vertrag beliebige Sachen vereinbaren, jedoch nicht gegen die Sitten verstoßen.
	\subsubsection{Pflichten}
	Der Arbeitgeber und -nehmer haben gegenseitige Pflichten: \\
	Für den Arbeitnehmer sind das:
	\begin{itemize}
		\item{Treuepflicht}
		\begin{itemize}
			\item{Der Arbeitnehmer muss die Interessen der Firma wahren.}
			\item{Konkurrenzverbot}
			\begin{itemize}
			\item{Ein Arbeitnehmer darf nicht zum Konkurrenten der Firma wechseln.}
			\item{Das Konkurrenzverbot tritt jedoch nur in Kraft, wenn der AN mehr als 4300€ Brutto verdient hat.}
			\end{itemize}
			\item{Ausbildungskostenrückersatz}
			\begin{itemize}
				\item{Wenn ein AN einen Kurs über den AG absolviert.}
				\item{Der AN muss eine Vereinbarung unterschrieben haben.}
				\item{Der Wert des AN muss am Markt gestiegen sein (Muss jedoch nicht zwingend mehr verdienen.)}
			\end{itemize}
		\end{itemize}
	\end{itemize}
	\subsection{Angestellte}
	Oft unterscheidet man zwischen Arbeitern und Angestellten. In der Regel ist ein Angestellter jemand im kaufmännischen Dienst und sonstigen höheren nicht-kaufmännischen Diensten, oder allgemeine Bürotätigkeiten leisten. Arbeiter hingegen werden nicht genau definiert und sind jene AN, welche größtenteils physische Tätigkeiten erbringen. Für die längste Zeit wurden Arbeiter in vielen Teilen des Arbeitsrechts benachteiligt und hatten so stets nur 14 Tage Kündigungsfrist, während abhängig von der Dienstzeit Angestellte bis zu 5 Monate genossen. Erst mit Beginn 2018 wurden Arbeiter und Angestellte gleichgestellt, wodurch die Kündigungsfrist und die Lohnfortzahlung im Krankheitsfall angeglichen wurde. 
	\subsection{Beschäftigungsverhältnis}
	\subsubsection{Dienstverhältnis}
	Ein Dienstverhältnis ist ein Dauerschuldverhältnis des Dienstnehmers. Es besteht jedoch keine Erfolgspflicht und schuldet nur das bloße Bemühen. Man wird dabei in eine fremde Betriebsorganisation eingegliedert und Arbeitsort, -zeit und -art sind vorgegeben. Jegliche Leistung ist persönlich zu erbringen.
	\subsubsection{Freies Dienstverhältnis}
	Ein freies Dienstverhältnis liegt vor, wenn sich jemand im wesentlichen persönlich aber ohne Vorliegen eines persönlichen Abhängigkeitsverhältnisses, auf bestimmte oder unbestimmte Zeit verpflichtet (Dauerschuldverhältnis). Also hat man wie im Dientvertrag eine Schuld zu begleichen, die Rahmenbedingungen sind jedoch bedeutend flexibler.
	\subsubsection{Werkverhältnis}
	Ein Werkvertrag hingegen ist ein Zielverhältnis und das relevante ist die getätigte Arbeit. Wie diese Arbeit vollrichtet wird ist irrelevant. Man könnte sogar jemand anderen anstellen diese Arbeit zu vollrichten. In einem Werkverhältnis könnte man sich auch vertreten lassen.
	\subsubsection{Lehrverhältnis}
	Ein Lehrverhältnis ist ein befristetes Dienstverhältnis und dient zur Ausbildung eines Lehrlings. Da ein Lehrverhältnis befristet ist, kann man nicht kündigen (oder gekündigt werden). Die Probezeit ist im Vergleich zu Angestellten (1 Monat) mit drei Monaten relativ lange. Während dieser Probezeit kann man das Verhältnis ohne Angabe von Gründen beenden. Ein Lehrverhältnis ist dual und man arbeitet so zusätzlich zu einer regulären Ausbildung.
	\subsection{Beendigung des Arbeitsverhältnisses}
	Ein Arbeitsverhältnis ist das Beenden des Arbeitsvertrages zwischen einem AN und dem AG. Diese können durch eine Vielzahl von Wegen geschehen:
	\begin{itemize}
		\item{Ende der Probezeit}
		\item{Befristetes Dienstverhältnis endet}
		\item{Kündigung}
		\item{Einvernehmliches Auflösen}
		\item{Entlassung}
		\item{Vorzeitiger Austritt}
		\item{Tod}
	\end{itemize}
	\subsubsection{Probezeit}
	Während der Probezeit kann das Dienstverhältnis von beider Seiten ohne Einhaltung von Fristen und Terminen oder Angabe von Gründen beendet werden. Die Probezeit dauert in der Regel 1 Monat, wobei es abhängig des Dienstverhältnisses auch 14 Tage sein kann. Lehrlinge haben speziell 3 Monate Probezeit, da Lehrlinge danach einen Kündigungsschutz genießen. \\
	Wenn man das Dienstverhältnis innerhalb der Probezeit beenden will, muss die Auflösungserklärung am letzten Tag der Probezeit in Händen des Arbeitnehmers sein und es reicht nicht, dass der Poststempel innerhalb der Frist lag.
	\subsubsection{Befristetes Dienstverhältnis}
	Ein befristetes Dienstverhältnis ist ein Dienstverhältnis mit einem Kalendermäßig fixierten Enddatum. Das Dienstverhältnis endet mit diesem Enddatum.
	\subsubsection{Kündigung}
	Bei der Kündigung wird ein unbefristetes Arbeitsverhältnis beendet. Sollte der Arbeitnehmer den Job aufgeben wollen, spricht man von einer Arbeitnehmerkündigung. Eine Kündigung ist in der Regel nur bei unbefristeten Arbeitsverhältnissen möglich (wobei laut OGH sehr lange Befristungen (5 Jahre+) auch kündigbar sind). \\
	Eine Kündigung ist eine einseitige, unwiderrufliche und empfangsbedürftige Willdenerklärung bei der der Vertragspartner über das Vorhaben informiert wird. 
	\begin{itemize}
		\item{Einseitig}
		\begin{itemize}
			\item{Da sie nur von einem Vertragspartner ausgesprochen werden muss}
		\end{itemize}
		\item{Unwiderruflich}
		\begin{itemize}
			\item{Man hat keinen Rechtsanspruch die Kündigung zurücknehmen zu können}
		\end{itemize}
		\item{Empfangsbedürftig}
		\begin{itemize}
			\item{Der Vertragspartner muss darüber informiert werden}
		\end{itemize}
	\end{itemize}
	Da das Gesetz keine bestimmte Kündigungsform vorsieht, kann man entweder schriftlich oder mündlich kündigen. \\
	Bei einer Arbeitgeberkündigung (Kündigung von Seiten des Arbeitgebers) hat der Arbeitnehmer Anspruch auf Postensuchtage. Diese ist eine Freizeit in der man sich auf die Suche nach einer neuen Arbeit macht und entspricht $\nicefrac{1}{5}$ der wöchentlichen Arbeitszeit (Also 7,7 Stunden bei einer Arbeitswoche von 38,5 Stunden oder ein Tag pro Woche). Dieser Tag muss davor mit dem Arbeitgeber vereinbart werden.
	\subparagraphlb{Kündigungsfristen}
	Bei der Kündigungsfrist unterscheidet man zwischen der Kündigungsfrist und dem Kündigungstermin. Dabei ist die Kündigungsfrist die Zeitspanne zwischen der Kündigungserklärung und dem Kündigungstermin, während der Kündigungstermin der Tag an dem das Dienstverhältnis beendet wird. Bei der Dauer der Kündigungsfrist muss man zwischen der Arbeitgeber- und Arbeitnehmerkündigung unterscheiden. Abhängig von der Länge des Dienstverhältnisses, ist die Kündigung auf Arbeitgeberseite länger befristet: \\
	\begin{tabular}{| l | l | l | l |}
		\toprule
		\multicolumn{4}{| c |}{Angestellte und Arbeiter} \\ \hline
		\multicolumn{2}{| c |}{Arbeitgeberkündigung} & \multicolumn{2}{| c |}{Arbeitnehmerkündigung} \\ \hline
		im 1. und 2. Dienstjahr & 6 Wochen & - & - \\ \hline
		im 3. und 5. Dienstjahr & 2 Monate & - & - \\ \hline
		im 6. bis 15. Dienstjahr & 3 Monate & Unabhängig von der Dienstdauer & 1 Monat \\ \hline
		im 16. bis 25. Dienstjahr & 4 Monate & - & - \\ \hline
		ab dem 26. Dienstjahr & 5 Monate & - & - \\
		\bottomrule
	\end{tabular}
	In der Regel ist der Kündigungstermin jeweils am Quartalsende (31.3., 30.6., 30.9. oder 31.12), wobei bei ausdrücklicher Vereinbarung auch der 15. oder letzte Tag des Monats zulässig ist (Abhängig vom KV). Somit stehen dem Arbeitgeber 24 Termine zur Verfügung: Jeweils der 15. oder letzte Tag des Monats oder einer der vier Quartalsenden.
	\paragraphlb{Kündigung bei Krankenstand}
	Während des Krankenstands oder bei Krankheit hat man weder einen Kündigungsschutz, noch ein Kündigungsverbot. Die Kündigung wird dabei auch durch den Krankenstand nicht verzögert und endet wie erwartet am Stichtag. Da man sich während des Krankenstands zuhause oder im Spital aufhalten sollte, ist eine Kündigung dort hin zu schicken. Sollte die Krankheit über die Dauer der Kündigungsfrist bestehen, muss der Arbeitgeber weiter das Krankengeld bezahlen. Darauf hat der Arbeitnehmer Anspruch bis er dem Sozialträger übertragen wird oder die Krankheit endet. Wenn man jedoch in den Krankenstand geht nachdem eine Kündigung ausgesprochen wurde, endet der Krankenstand mit Ende des Dienstverhältnisses
	\paragraphlb{Kündigung im Urlaub}
	Man kann auch im Urlaub gekündigt werden. Da diese jedoch empfangsbedürftig ist, muss die Kündigung an den Arbeitnehmer übertragen werden. Die Kündigung beginnt danach erst, sobald die Kündigung gelesen wurde oder, falls sie an die Wohnadresse gesendet wurde, am Tag der Rückkehr.
	\subparagraphlb{Kündigungsentschädigung}
	Ein Arbeitnehmer hat eine
	\subsubsection{Einvernehmliches Auflösen}
	Ein einvernehmliches Auflösen geschieht, wenn das Beenden des Verhältnisses im Wille beider Vereinbarungspartner steht. Dabei muss man sich über einen bestimmten Tag einigen. Während dies wie die meisten Dinge im Arbeitsrecht schriftlich, mündlich oder stillschweigend geschehen kann, ist es dringend zu empfehlen ein schriftliches Abkommen zu verlangen. \\
	Falls ein Betriebsrat existiert, kann der Arbeitnehmer eine Unterredung mit dem Beitrebsrat verlangen. Falls dies geschieht kann die Auflösung erst zwei Arbeitstage danach geschehen. \\
	Das einvernehmliche Auflösen ist auch für geschützte Personen wie werdende Mütter, Wehrpflichtige oder Karenztragende möglich, dies ist jedoch nur in schriftlicher Form möglich. Wenn es sich um einen Präsenz-, Ausbildungs- oder Zivildiener bzw. einem minderjährigen Arbeitnehmer oder Lehrling handelt, muss zuvor eine Rechtsbelehrung beim Arbeits- oder Sozialgericht geschehen, in der klargestellt wird, dass bekannt ist, dass ein geschützter Arbeitsplatz verlassen wird. Bei minderjährigen muss zusätzlich die Zustimmung des Erziehungsberechtigten erfolgen. \\
	Bei Auflösung hat man Anspruch auf Gehalt bis zum Ende des Arbeitsverhältnisses sowie ausständiges Winter- und Urlaubsgeld. Bei einer einvernehmlichen Auflösung hat man kein Anrecht auf Postensuchtage.
	\subsubsection{Entlassung}
	Eine Entlassung ist die begründete sofortige Auflösung des Dienstverhältnisses vonseiten des Arbeitgebers. Gründe für eine Entlassung können sein:
	\begin{itemize}
		\item{Dienstunfähigkeit}
		\item{Trunksucht und wiederholt ohne Erfolg ermahnt}
		\item{Diebstahl, Veruntreuung oder sonstige Vertrauensunwürdigkeit}
		\item{Ehrenbeleidigung, Körperverletzung oder anderes}
		\item{Verbüßung von längeren Freiheitsstrafen}
		\item{Verweigerung der Arbeit}
		\item{Unbefugtes Verlassen der Arbeit oder beharrliche vernachlässigung der Pflichten}
		\item{Nichtverwendung von bereitgestellten Schutzvorrichtungen}
		\begin{itemize}
			\item{Wenn zum Beispiel ein Gebäudereiniger sich weigert die zur Verfügung gestellten Geräte zu verwenden}
		\end{itemize}
	\end{itemize}
	\subsubsection{Austritt}
	Ein Austritt ist die sofortige begründete Auflösung des Arbeitsverhältnisses vonseiten des Arbeitnehmers. Die Beweislast liegt hierbei jedoch bei dem Arbeitnehmer. Gründe können sein:
	\begin{itemize}
		\item{Bei Dienstunfähigkeit aufgrund der Arbeit (Berufskrankheit oder gesundheitliche Schäden aufgrund der Arbeit)}
		\item{ungebührliche Vorenthaltung des Lohns}
		\item{Ehrenbeleidigung, Körperverletzung}
		\item{Keine Bereitstellungvon notwendigen Schutzmaßnahmen}
	\end{itemize}
	\subsubsection{Außerordentliche Auflösung des Lehrverhältnis}
	Seit 2008 ist es möglich bei Lehrverhältnissen dieses aufzulösen. Der Termin dafür ist der letzte Tag des 12. Monats oder bei Lehrberufen mit einer Lehrzeit von 3 Jahren oder mehr mit Ablauf des 24. Monats. Anderenfalls wird ads Verhältnis mit Ende des Monats beendet.
	\subsubsection{Abfertigung}
	Die Abfertigung ist eine Auszahlung die nach Beendigung des Arbeitsverhältnisses an den Arbeitnehmer zu entrichten ist. Dabei gibt es zwei Arten der Abfertigung: Die Abfertigung (alt) und Abfertigung (neu).
	\paragraphlb{Abfertigung (alt)} 
	Die alte Abfertigung war abhängig von der Dienstzeit und entsprach einer gewissen Menge an Monatsgehältern.  So erhielt man nach einer gewissen Menge an Dienstjahren 10 Monate an Abfertigung. Da das für einen Arbeitgeber oft einen großen finanziellen Aufwand bedeutete, waren diese sehr darauf aus, dass der Arbeitnehmer selbst kündigt, da man so die Abfertigung nicht zahlen musste. 
	\paragraphlb{Abfertigung (neu)}
	Um diese Strategie zu verhindern, wurden alle neuen Abfertigungen auf die neue Regelung umgestellt. Hierbei zahlt der Arbeitgeber monatlich einen gewissen Betrag auf ein Konto ein. Dieses Geld steht dem Arbeitnehmer stets zur Verfügung und verschwindet nicht, wenn man den Arbeitgeber wechselt.
	\subsubsection{Zusammenfassung}
	Im Allgemeinen kann man zwischen vier Arten der Beendigung sprechen: \\
	\begin{tabular}{| c | c | c |}
		\toprule
		\makecell{mit Zustimmung des \\ Vertragspartners} & \multicolumn{2}{| c |}{\makecell{ohne Zustimmung des Vertragspartners}} \\ \midrule
		einvernehmliche Auflösung & ohne Gründe & mit Gründen \\ \hline
		\makecell{Beschränkungen für werdende \\ Mütter, Wehrdienstpflichtige \\ oder Lehrlinge} & Fristablauf & \makecell{Entlassung \\ (vonseiten des Arbeitgebers)} \\ \hline
		 & Kündigung & \makecell{Austritt \\ (vonseiten des Arbeitnehmers)} \\
		\bottomrule
	\end{tabular}
	\subsection{Arbeitszeit}
	Die Arbeitszeit ist der Beginn bis zum Ende der Arbeitszeit ohne Ruhepause. Das österreichische Arbeitsrecht bescreibt ein maximum von 40 Arbeitsstunden pro Woche, jedoch sehen die meisten Kollektivverträge 38,5 Stunden vor.
	Es gibt drei Arbeitszeitenmodelle:
	\begin{itemize}
		\item{Fixe Arbeitszeit}
		\begin{itemize}
			\item{Der traditionelle 8 to 5 wobei man genau zwischen dieser Zeit zu arbeiten hat.}
		\end{itemize}
		\item{Gleitzeit}
		\begin{itemize}
			\item{Während Kernzeiten existieren (Zum Beispiel von 8 bis 12) an denen man anwesend sein muss, kann man sich die Anfangs- und Endzeit frei wählen. Man muss jedoch trotzdem sein monatliches Pensum erfüllen. (Flexibel für den Arbeitsnehmer)}
		\end{itemize}
		\item{Flexible Arbeitszeit}
		\begin{itemize}
			\item{Während man über einen Monat trotzdem die vollen Stunden arbeitet, sind die Stunden einzelner Tage flexibel, idealerweise an den Bedarf des Arbeitgebers angepasst. (Flexibel für den Arbeitgeber)}
			\item{Die flexible Arbeitszeit wurde 2018 eingeführt um die Bedürfnisse der Arbeitgeber besser zu treffen. So konnte man statt der vorherigen 8 Stunden 4 Stunden Überstunden machen (wobei die 11. und 12. Stunde "optional" ist, was jedoch aufgrund Druck des Arbeitgebers oft nicht der Fall ist)}
			\item{Da die Ruhezeiten jedoch dadurch ein Problem werden können (Wenn zum Beispiel ein Koch um 11 aufhört und um 8 wieder beginnen soll geht das durch die 11 Stunden Ruhezeit nicht), wurde die Ruhezeit in der Gastronomie auf 8 Stunden verringert.} 
		\end{itemize}
	\end{itemize}
	\subsubsection{Fenstertage}
	Fenstertage sind Arbeitstage, welche zwischen einem Feiertag und dem Wochenende liegen. Diese Tage werden gerne für ein verlängertes Wochenende freigenommen. Es ist normalerweise nicht möglich einen einzelnen Urlaubstag zu nehmen (Was jedoch durch das Günstigkeitsprinzip gedeckt ist), weshalb man einvernehmlich einen Fenstertag einarbeiten kann. Dabei wird dieser 1:1 abgegolten. Das hat für beide Parteien einen Vorteil, da der Arbeitgeber sich dadurch Überstunden spart und der Arbeitnehmer ein verlängertes Wochenende bekommt ohne Urlaub aufgeben zu müssen.
	\subsubsection{Überstunden und Mehrstunden}
	Überstunden sind Arbeitsstunden, welche über die Normalarbeitszeit fallen, und mehr als 1:1 abgegolten werden. So können Überstunden mit 50\% oder 100\% extra abgegolten werden wodurch für 60 Minuten Arbeitszeit 90 bzw. 120 Minuten verrechnet werden. \\
	Mehrstunden auf der anderen Seite sind Stunden, welche zwar über die Normalarbeitszeit fallen, jedoch nicht besser vergolten werden. Das kann passieren, wenn man bei flexibler Arbeitszeit 10 Stunden oder wenige arbeitet, solange dies die monatliche Arbeitszeit nicht überschreitet.
	\paragraphlb{Überstundenverbot}
	Gleich wie beim Kündigungsschutz, unterliegen werdende mütte sowie Jugendliche unter 16 Jahren einem Überstundneverbot.
	\paragraphlb{Bezahlung}
	Laut Gesetz sollten Überstunden ausbezahlt werden, und Zeitausgleich wird nur nach Absprache gewährt. In der Regel findet jedoch das Gegenteil statt und es wird oft lieber Zeitausgleich gewährt.
	\paragraphlb{Ruhepausen}
	Bei mehr als 6 Stunden Tagesarbeitszeit muss ein Mal eine Pause von 30 Minuten gewährt werden.
	\subsection{Urlaub}
	Urlaub



















  
\end{document}