\documentclass{article}

\usepackage{geometry}
\usepackage{makecell}
\usepackage{array}
\usepackage{multicol}
\usepackage{setspace}
\usepackage{changepage}
\usepackage{booktabs}
\usepackage[explicit]{titlesec}
\usepackage{hyperref}
\usepackage{graphicx}
\usepackage{cprotect}
\usepackage{float}
\newcolumntype{?}{!{\vrule width 1pt}}
\newcommand{\paragraphlb}[1]{\paragraph{#1}\mbox{}\\}
\newcommand{\subparagraphlb}[1]{\subparagraph{#1}\mbox{}\\}
\renewcommand{\contentsname}{Inhaltsverzeichnis:}
\renewcommand\theadalign{tl}
\setstretch{1.10}
\setlength{\parindent}{0pt}
\setcounter{tocdepth}{5}

\titleformat{\section}
  {\normalfont\Large\bfseries}{\thesection}{1em}{\hyperlink{sec-\thesection}{#1}
\addtocontents{toc}{\protect\hypertarget{sec-\thesection}{}}}
\titleformat{name=\section,numberless}
  {\normalfont\Large\bfseries}{}{0pt}{#1}

\titleformat{\subsection}
  {\normalfont\large\bfseries}{\thesubsection}{1em}{\hyperlink{subsec-\thesubsection}{#1}
\addtocontents{toc}{\protect\hypertarget{subsec-\thesubsection}{}}}
\titleformat{name=\subsection,numberless}
  {\normalfont\large\bfseries}{\thesubsection}{0pt}{#1}

\hypersetup{
    colorlinks,
    citecolor=black,
    filecolor=black,
    linkcolor=black,
    urlcolor=black
}

\geometry{top=12mm, left=1cm, right=2cm}
\title{\vspace{-1cm}Controlling 1}
\author{Andreas Hofer}

\begin{document}
	\maketitle
	\tableofcontents
	\section{Kostenrechnung}
	Es gibt vier Ebenen des betrieblichen Rechnungswesens. Drei davon gehören zum externen Rechnungswesen, also der Finanzbuchhaltung. Dabei wird die Auszahlung der Einzahlung gegenübergestellt
	\subsection{Kostenrechnungssysteme}
	Ein Kostenrechnungssystem ist ein System zur Auswertung der Kosten innerhalb des Unternehmens. Dabei unterliegt dessen Einführung keine rechtliche Vorgabe, weshalb sie im Ermessen des Unternehmens liegt. Formale Kriterien sind: Wirtschaftlichkeit, Relevanz, besser Schnell als Genau, Häufigkeit, Flexibilität, Zeitliche und sprachliche Entsprechung. \\
	Generell soll die Kostenrechnung folgende Fragen beantworten:
	\begin{itemize}
		\item{Welchen Rechnungsobjekte sollen die Kosten ermittelt werden?}
		\item{Nach welchem Grundprinzip soll die Kostenverrechnung erfolgen?}
		\item{Soll es der Teil- oder der Vollkostenrechnung folgen?}
		\item{Wie groß soll der Zeitbezug und Verrechnungsumfang sein.}
	\end{itemize}
	\paragraphlb{Externe Sicht}
	Eine effektive Kostenrechnung ist aus externer und interner Sicht relevant. Durch Wettbewerb sind Unternehmen ständig angehalten ihre Kosten zureduzieren um wettbewerbsfähig zu bleiben. Es ist auch nötig zur Auswertung einer möglichen ANgebotserweiterung, da man so herausfinden kann, ob es sich lohnt.
	\paragraphlb{Interne Sicht}
	Aus interner Sicht ist die Genauigkeit und Schnelligkeit der Kalkulationen relevant, welche dadurch verbessert werden können. Wenn man ein "Profit Center" einführen will, muss man auch wissen wo diese liegen.
	\subsubsection{Verursacherprinzip}
	Die dominierende Regel in der Kostenrechnung ist das Verursacherprinzip und besagt, dass einem einzelnen Kostenträger nur jene Kostne zugerechnet werden sollten, welcher sie auch verursacht. Das ist idealerweise die beste Lösung, oft jedoch nicht einwandfrei möglich, da man nicht immer sagen kann wer genau welche Kosten verursacht hat.
	\paragraphlb{Durchschnittsprinzip}
	Als realitätsnähere Variante besteht das Durchschnittsprinzip, bei welchem Kosten im Durschnitt auf ihre zugehörigen Träger.
	\subsubsection{Tragfähigkeitsprinzip}
	Als Gegenstück dazu besteht das Tragfähigkeitsprinzip, nach dem Kosten dem Träger anhand ihrer Möglichkeit diese zu tragen zuzordnen. Also wurden einige "High-Performer" im Unternehmen mehr Kosten tragen als sie wirklich verursachen.
	\subsubsection{Kostenwürfel}
	Der Kostenwürfel von DEHYLE beschreibt die Dimensionen von Kosten anhand eines Würfels. Laut des Würfels gibt es drei Dimensionen von Kosten:
	\begin{itemize}
		\item{Struktur}
		\begin{itemize}
			\item{Variable Kosten - Kosten pro erzeugtem Produkt}
			\item{Fixkosten - Kosten für die Infrastruktur etc}
		\end{itemize}
		\item{Beeinflussbarkeit}
		\begin{itemize}
			\item{Kurzfristige Kosten - Kosten welche innerhalb eines Monats verändert werden können}
			\item{Langfristige Kosten - Kosten welche länger als einen Monat benötigen um verändert zu werden.}
		\end{itemize}
		\item{Zurechenbarkeit}
		\begin{itemize}
			\item{Einzelkosten - Kosten welche einem spezifischen Produkt zugeordnet werden}
			\item{Gemeinkosten - Kosten welche nicht direkt zugeordnet werden können und deshalb auf die einzelnen Positionen umgelegt werden müssen}
		\end{itemize}
	\end{itemize}
	\subsubsection{Kostenverläufe}
	Die Kostenverläufe sind verschiedene Kurvenmodellierungen der Kostenfunktion. 
	\begin{itemize}
		\item{Proportionale Kosten}
		\begin{itemize}
			\item{Für jede Stückzahl steigt der gleiche Wert an}
		\end{itemize}
		\item{Degressive Kosten}
		\begin{itemize}
			\item{Die Kosten pro Stück sinken mit steigender Anzahl}
		\end{itemize}
		\item{Progressive Kosten}
		\begin{itemize}
			\item{Die Kosten pro Stück steigen mit steigender Anzahl}
		\end{itemize}
		\item{Regressive Kosten}
		\begin{itemize}
			\item{Die Gesamtkosten sinken bei steigender Stückzahl}
		\end{itemize}
	\end{itemize}
	\subsubsection{Gliederung von Kostenrechngungssystemen}
	KRS kann man anhand einer Zeit- oder Strukturkomponente kategorisieren:
	\begin{itemize}
		\item{Zeitkomponente}
		\begin{itemize}
			\item{Istkostenrechnung}
			\begin{itemize}
				\item{Kosten für den tatsächlichen Verbrauch und sonstigem Einsatz}
			\end{itemize}
			\item{Normalkostenrechnung}
			\begin{itemize}
				\item{Der Durchschnitt der Istkosten mehrerer Abrechnungsperioden}
			\end{itemize}
			\item{Planostenrechnung}
			\begin{itemize}
				\item{Kosten bei geplanter Auslastung und Beschäftigung abhängig von der momentanen wirtschaftlichen Leistung}
			\end{itemize}
		\end{itemize}
		\item{Umfangskomponente}
		\begin{itemize}
			\item{Vollkostenrechnung}
			\begin{itemize}
				\item{Verrechnet variable und fixe Kosten}
			\end{itemize}
			\item{Teilkostenrechnung}
			\begin{itemize}
				\item{}
			\end{itemize}
		\end{itemize}
	\end{itemize}
	\section{Controlling}
	Controlling ist die Managementunterstützung für Kostenrechnung. So ist Controlling für Zielsetzung, Planung, Steuerung und Kontrolle verantwortlich. Zielsetzung ist dabei jedoch nicht die Zielplanung sondern nur die Unterstützung der Prozesse. Der Controller hat keine direkte Kontrolle, ein Controller steuert und überwacht, die Entscheidung wird jedoch von jemand anderes getroffen.
	\subsection{Management und Führung}
	Der Unterschied zwischen Management und Führung ist, dass die Führung die Richtung des Unternehmens entscheidet, und das Management den Ablauf dieser Richtung. In der Regel sind Managementfehler auch bedeutend schneller ersichtlich, da Probleme in einem Prozess besser erkennbar sind als in der Strategie. So können Führungsfehler zumal erst nach 5 Jahren ersichtlich werden. In der Regel wird Management und Führung jedoch Synonym verwendet. \\
	Controlling dient der Unterstützung des Managements. Während Entscheidungsträger das Unternehmen führen und Mitarbeiter motivieren sollen, sollen Controller das Unternehmen objektiver darstellen und diesen Sachverhalt dem Management vorzulegen.
	Durch Aufschlüsselung der Kosten kann man die Struktur der Kosten- und Leistungsarten analysieren oder innerhalb des Unternehmens vergleichen. Zusätzlich kann es sein, dass man für ein anderes Unternehmen eine Leistung vollbringt welche dadurch abgerechnet werden kann. Idealerweise sollten die Arten dabei erfassen, ob:
	\begin{itemize}
		\item{Eindeutigkeit}
		\begin{itemize}
			\item{Kosten sollten so eindeutig wie möglich zugeordnet werden.}
		\end{itemize}
		\item{Einheitlichkeit}
		\begin{itemize}
			\item{Kosten sollten wenn möglich stets der gleichen Art zugeordnet werden}
		\end{itemize}
		\item{Vollständigkeit}
		\item{Wirtschaftlichkeit}
	\end{itemize}
	\subsection{Betriebsüberleitung}
	Die Betriebsüberleitung ist der Übergang von Gewinn- und Verlustrechnungen aus dem externen Rechnungswesen in das interne Rechnungswesen. Dabei werden Aufwendungen zu Kosten und Erträge zu Leistungen übergeleitet. Diese Überleitung ist ein relativ intensiver Prozess, weshalb sie von kleineren Unternehmen oft ausgelassen wird. Eine Umfrage ergab, dass 65\% aller Unternehmen diese Überleitung durchführen, anstatt sie direkt zu übernehmen. \\
	Einige Erträge und Aufwendungen, welche im externen Rechnungswesen relevant sind, sind intern nicht relevant. Diese nennt man neutrale Erträge. Allgemein sind die drei Kriterien, ob sie Betriebsfremd, Außerordentlich oder Periodenfremd sind. \\
	Der Übergang erfolgt in 4 Schritten:
	\begin{enumerate}
		\item{Zuerst werden neutrale Aufwände verworfen}
		\item{Danach wird}
	\end{enumerate}
	\subsubsection{Kostenartenrechnung}
	Die Kostenartenrechnung spaltet Kosten in ihre Träger auf. Alles was in einer Weise Kosten erzeugt, wird dabei klassifiziert. Die klassischen Gruppen der Kostenarten sind:
	\begin{itemize}
		\item{Material}
		\item{Personal}
		\item{Energie}
		\item{Steuern, Beiträge und Versicherungen}
		\item{Kalkulatorische Kosten}
	\end{itemize}
	\paragraphlb{Kalkulatorische Kosten}
	Kalkulatorische Kosten sind Kosten die keinem Aufwand gegenüberstehen.
	\subparagraphlb{Abschreibung}
	Die Abschreibung dient um den Wertverlust von Geräten entgegenzurechnen. Dabei unterscheidet man zwischen der Bilanziellen und der Kalkulatorischen Abschreibung. Während die bilanzielle Abschreibung jedes Jahr einen Teil der Kosten abzieht (abhängig von der Länge der Nutzung), berechnet die kalkualtorische Abschreibung zuerst den Wiederbeschaffungswert (WW), welcher einer jährlichen Preissteigerung entspricht und dieses wird dann minus dem Restwert jährlich abgezogen. Nach Ende der Abschreibungsdauer hat man dann auch schon die Kosten für die Beschaffung eines neuen Geräts abgeschrieben. \\
	Eine Sonderregel bei Abschreibungen sind geringwertige Gürter, in der Regel weniger als 1000€, welche sofort dieses Jahr vollständig abgeschrieben werden können.
	\subparagraphlb{Wagnis}
	Ein Wagnis ist eine Abschätzung der Gefahr des Misslingens. Das können allgemeine Unternehmerwagnisse wie eine Veränderung der Gesamtwirtschaft oder einem speziellen Einzelwagnis, welches eine gewisse Anlage oder ein Bestand sein können. \\
	Während ein allgemeines Unternehmenswagnis im Gewinn abgebildet wird, müssen spezielle Einzelwagnisse in der Kostenstellenrechnung abgebildet werden.
	\subparagraphlb{Kalkulatorischer Unternehmerlohn}
	Angestellte in einer Kapitalgesellschaft werden mit einem Lohn entschädigt und dieses wird auch in der Kostenrechnung abgebildet. Bei Personengesellschaften und Mitarbeitern welche mitarbeiten, aber aus dem Ertrag der Gesellschaft ihr Gehalt beziehen müssen extra abgebildet werden. Das gleiche gilt auch für kalkulatorische Miete.
	\subparagraphlb{Kalkulatorische Zinsen}
	Bei der zeitweiligen Überlassung von Kapital entstehen für den Verleiher Opportunitätskosten, also Kosten, welche dem Verleiher entstehen, weil etwas anderes mit dem Kapital geschehen hätte können. \\
	Für das Anlagevermögen nimmt man in der Regel die Hälfte des Quotienten von Anlage- und Restwert multipliziert mit der Laufdauer.
	Der Zinssatz kann schwer zu bestimmen sein, wird in der Regel jedoch durch das Verhältnis von Eigen- zu Gesamtkapital und dem Verhältnis von Fremd- zum Gesamtkapital gesehen.
	\subsection{Kostenstellenrechnung}
	Gemeinkosten, bestimmt in der Kostenartenrechnung, müssen in Kostenstellen abgerechnet werden (Zum Beispiel eine Materialkostenstelle). Eine Kostenstelle kann ein Betrieb, ein Projekt oder auch eine Person sein. Sie fragt an welcher Stelle die Kosten angefallen sind oder wo Kosten anfallen werden. So können nicht direkt zurechenbare Kosten stattdessen auf Kostensträger umgeschlagen werden. Grundsätzlich wird eine Kostenstelle anhand von drei Grundsätzen definiert:
	\begin{itemize}
		\item{Eine Kostenstelle muss ein selbstständiger Verantowrtungsbereich sein}
		\item{Eine Kostenstelle muss möglichst genaue Maßgrößen besitzen}
		\item{Auf eine Kostenstelle müssen Kostenbelege genau und gleichzeitig verbucht werden können}
	\end{itemize}
	Bei Kostenstellen wird nach Leistungsbereichen und Abrechnungsarten gegliedert. Leistungsbereiche werden nach Fertigungsstellen, Hallen, Werken oder Verantwortungsbereiche gegliedert. Abrechnungsarten sind entweder Vordkosten- oder Endkostenstellen und müssen entweder weiterverrechnet oder in die Kostensträgerrechnung eingebunden werden. \\
	\paragraphlb{Vorkostenstelle}
	Die Vorkostenstelle ist nicht unmittelbar mit der Fertigung verbunden und wird deshalb an die innerbetriebliche Leistungsverrechnung weiterverrechnet.
	\paragraphlb{Endekostenstelle}
	Endkostenstellen gehören nicht zum betrieblichen Hauptzweck (Wie eine Druckerei) oder dienen der direkten Erstellung von Marktleistungen.
	\paragraphlb{Betriebsabrechnungsbogen (BAB)}
	Der Betriebabrechnungsbogen (BAB) dient zur Zuteilung und läuft heutzutage meist vollautomatisiert ab. Dieser wird anhand eines Kostenstellenplans abgerechnet, welcher die verschiedenen Kostenstellen definiert. \\
	Die Betriebsabrechnung verläuft nach einem strikten Plan:
	\begin{enumerate}
		\item{Die Kostenstellen gemäß des Kostenstellenplans übernommen.}
		\item{Die Kostenarten werden}
	\end{enumerate}
	Danach muss man Zuschlagsplätze definieren. Dabei werden den Kostenträgern die Gemeinkosten zugeschlagen. Hiefür muss man eine Bezugrsgröße definiren (Z.B. Maschinenstunden oder Gigabyte). Danach werden die Gemeinkosten der Kostenstelle durch die Bezugsgröße dividiert und der Kostenstelle zugerechnet.
	
	






	























  
\end{document}