\documentclass{article}

\usepackage{geometry}
\usepackage{makecell}
\usepackage{array}
\usepackage{multicol}
\usepackage{setspace}
\usepackage{changepage}
\usepackage{booktabs}
\usepackage[explicit]{titlesec}
\usepackage{hyperref}
\usepackage{graphicx}
\usepackage{cprotect}
\usepackage{float}
\newcolumntype{?}{!{\vrule width 1pt}}
\newcommand{\paragraphlb}[1]{\paragraph{#1}\mbox{}\\}
\renewcommand{\contentsname}{Inhaltsverzeichnis:}
\renewcommand\theadalign{tl}
\setstretch{1.10}
\setlength{\parindent}{0pt}

\titleformat{\section}
  {\normalfont\Large\bfseries}{\thesection}{1em}{\hyperlink{sec-\thesection}{#1}
\addtocontents{toc}{\protect\hypertarget{sec-\thesection}{}}}
\titleformat{name=\section,numberless}
  {\normalfont\Large\bfseries}{}{0pt}{#1}

\titleformat{\subsection}
  {\normalfont\large\bfseries}{\thesubsection}{1em}{\hyperlink{subsec-\thesubsection}{#1}
\addtocontents{toc}{\protect\hypertarget{subsec-\thesubsection}{}}}
\titleformat{name=\subsection,numberless}
  {\normalfont\large\bfseries}{\thesubsection}{0pt}{#1}

\hypersetup{
    colorlinks,
    citecolor=black,
    filecolor=black,
    linkcolor=black,
    urlcolor=black
}

\geometry{top=12mm, left=1cm, right=2cm}
\title{\vspace{-1cm}Netzwerktechnologien Advanced - Übung}
\author{Andreas Hofer}

\begin{document}
	\maketitle
	\tableofcontents
	\section{Heimnetzwerk}
	\section{OSI Referenzmodel}
	\section{Netzwerktechnik Grundbegriffe \& Tools}
	\section{Packet Tracer erste Übung}
	\section{Verschiedene Server im lokalen Netz}
	\section{Netzwerksniff mit Wireshark}
	\section{Abschlussübung Transport Layer}
	\section{Verkabelung}
	\section{Switch Einführung}
	\section{VLANs am Switch}
	\section{Subnetting und Routing Vertiefung}
	\begin{center}
	\textbf{Bei dieser Übung solltest du ein Subnet aufbauen und ein etwas größeres Netzwerk im Packet Tracer aufbauen.}
	\subsection{Subnetting}
	Wähle zuerst ein beliebiges /24 Netzwerk aus dem RFC1918 Bereichen.
	\textbf{Schreibe hier dein Netzwerk hin: 192.168.1.0/24}
	\begin{enumerate}
		\item{Subnetz 1}
		\begin{itemize}
			\item[]{Netzwerkadresse: 192.168.1.0/26}
			\item[]{Subnetzmaske: 255.255.192.0}
		\end{itemize}
	\end{enumerate}
	\end{center}
	\section{Routing Vertiefung}
	\section{ARP \& ICMP Übungen}
	\section{DHCP}
	\section{Router}
	\section{DHCP Konfiguration}
	\section{Inter VLAN Routing}
	\section{Routing}
	\section{Umbau auf Inter VLAN Routing mit Trunk}
	\section{Wiederholung der Grundlagen}
	\section{Layer 2 PDU}
	\section{ARP}
	\section{Switches und VLANs}
	\section{VTP \& STP}
	\section{Router Konfiguration \& IPv6}
	\section{Statisches Routing in einem größeren Netzwerk}
	\section{Dynmaic Routing}
	\section{BPG - Border Gateway Protocol}
	\section{Access Control Lists Introduction}
	























  
\end{document}