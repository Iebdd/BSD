\documentclass{article}

\usepackage{geometry}
\usepackage{makecell}
\usepackage{array}
\usepackage{multicol}
\usepackage{setspace}
\usepackage{changepage}
\usepackage{booktabs}
\usepackage[explicit]{titlesec}
\usepackage{hyperref}
\usepackage{graphicx}
\usepackage{cprotect}
\usepackage{float}
\newcolumntype{?}{!{\vrule width 1pt}}
\newcommand{\paragraphlb}[1]{\paragraph{#1}\mbox{}\\}
\newcommand{\subparagraphlb}[1]{\subparagraph{#1}\mbox{}\\}
\renewcommand{\contentsname}{Inhaltsverzeichnis:}
\renewcommand\theadalign{tl}
\setstretch{1.10}
\setlength{\parindent}{0pt}

\titleformat{\section}
  {\normalfont\Large\bfseries}{\thesection}{1em}{\hyperlink{sec-\thesection}{#1}
\addtocontents{toc}{\protect\hypertarget{sec-\thesection}{}}}
\titleformat{name=\section,numberless}
  {\normalfont\Large\bfseries}{}{0pt}{#1}

\titleformat{\subsection}
  {\normalfont\large\bfseries}{\thesubsection}{1em}{\hyperlink{subsec-\thesubsection}{#1}
\addtocontents{toc}{\protect\hypertarget{subsec-\thesubsection}{}}}
\titleformat{name=\subsection,numberless}
  {\normalfont\large\bfseries}{\thesubsection}{0pt}{#1}

\hypersetup{
    colorlinks,
    citecolor=black,
    filecolor=black,
    linkcolor=black,
    urlcolor=black
}

\geometry{top=12mm, left=1cm, right=2cm}
\title{\vspace{-1cm}Digitale Geschäftsmodelle 1}
\author{Andreas Hofer}

\begin{document}
	\maketitle
	\tableofcontents
	\section{Impulsvortrag}
	Dieser Teil bietet einen allgemeinen Einblick in die zwei Teilthemen
	\subsection{Digitale Netzökonomie}
	Die digitale Netzökonomie wurde sowohl technisch als auch sozial geprägt. Ein relevanter Begriff ist dabei das E-Business, welches die Verwendung von einem Geschäftsmodell bezeichnet und dabei Internettechnologien verwendet. Dabei wird auch die Beziehung zwischen den Geschäftspartnersn innerhalb der Kette von Wertschöpfungen miteinbezogen. Wichtig ist dabei, dass die Internettechnologien keine neuen Regeln definieren, sondern:
	\begin{itemize}
		\item{Regeln neu kombiniert}
		\item{einzelne Regeln hervorgehoben}
		\item{Erfolgsfaktoren und Perspektiven neu aufgestellt}
	\end{itemize}
	Die Digitale Netzökonomie überschneidet sich mit anderen, ähnlichen Begriffen und werden meist synonym verwendet. Einige dieser Begriffe sind:
	\begin{itemize}
		\item{Digitale Netzökonomie}
		\begin{itemize}
			\item{Die Mikroökonomischen Vorgänge beim Übergang von einer Industrie- zu einer Informationsgesellschaft}
		\end{itemize}
		\item{Neue Ökonomie}
		\begin{itemize}
			\item{Die Zeitspanne nach 1995, wobei niedrige Inflation einem großen Wirtschaftswachstum gegenüberstand}
		\end{itemize}
		\item{Digitale Ökonomie}
		\begin{itemize}
			\item{Globale Kooperation was zu kurzen Innovationszyklen führt}
		\end{itemize}
		\item{Internetökonomie}
		\begin{itemize}
			\item{Wandel in den 1990er Jahren aufgrund der Vernetzung der Welt und des Internets}
		\end{itemize}
	\end{itemize}
	\subsubsection{Prinzipien}
	Einige Prinzipien der digitalen Netzökonomie sind:
	\begin{itemize}
		\item{Senkung der Kosten}
		\begin{itemize}
			\item{Aufgrund nicht-linearer Skalierbarkeit von digitalen Inhalten, kann man bei hohen Nutzerzahlen und nur geringfügig größeren Kosten, eine höhere Marge erzielen}
		\end{itemize}
		\item{Dematerialisierung durch Digitalisierung}
		\begin{itemize}
			\item{Prozesse die physisch geschehen passieren danach digital. Wenn man zum Beispiel Wohnungen kurzfristig vermietet, kann man digital Zugriff erteilen und muss nicht physisch einen Schlüssel überreichen.}			
		\end{itemize}
		\item{Economies of Scale and Scope}
		\begin{itemize}
			\item{Auf deutsch Skaleneffekt und Verbundeffekt}
				\item{Der Skaleneffekt besagt, dass die Mehrkosten pro zusätzlichem Benutzer nicht linear steigen, wodurch bei höheren Nutzerzahlen ein größerer Gewinn erwirtschaftet werden kann.}
				\item{Der Verbundeffekt besagt, dass durch die Produktion viele ähnlicher Produkte, Prozesse relativ einfach angepasst werden können wodurch bei Produktion vieler Variationen keine linear höheren Kosten entstehen.}
		\end{itemize}
		\item{Neue Preis- und Erlösmodelle}
		\begin{itemize}
			\item{Anstatt ein Produkt ein Mal zu kaufen, gibt es vermehrt Abomodelle, wobei man einen monatlichen Betrag bezahlt und dadurch zwar mehr zahlt aber auch stets die neueste Version erhält.}
		\end{itemize}
		\item{Neue Rolle des Kunden}
		\begin{itemize}
			\item{Der Kunde kann sich heute viel besser über Produkte informieren, da es hunderte Kanäle und Websites gibt die Produkte vergleichen.}
			\item{Gleichzeitig können Kunden ihren Unmut über Produkte direkt mitteilen, welche auch von anderen eingesehen werden können. Dadurch können Kunden auch eine Marketingrolle einnehmen, da gute Produkte bessere Bewertungen haben.}
		\end{itemize}
		\subsubsection{Lock-In Effekt}
		Der Lock-In Effekt führt durch gewisse Ursachen dazu, dass man in einem Ökosystem oder einem Produkt 'gefangen' ist und nicht wechseln will oder kann. Das beste Beispiel dafür ist der Netzeffekt.
		\paragraphlb{Netzeffekt}
		In einem sozialen Netzwerk sind das wertvollste Gut die Personen, welche zu einer Vernetzung dieser führen. Ein Beispiel dafür ist Facebook, welches nach eigenen Angaben mehr als 1 Milliarde Benutzer hat. Die große Menge an Benutzern auf dieser Plattform führt dazu, dass dieses an Wert steigt, was wiederum neue Benutzer anzieht. Das nennt man einen \textbf{direkten positiven Netzeffekt}, da es so durch seinen Wert aktiv weiter an Wert gewinnt, indem neue Benutzer angezogen werden. Dazu steht der \textbf{indirekte positive Netzeffekt}, wobei der Wert der Plattform nicht direkt durch dessen Marktmacht entsteht sondern indirekt von Diensten die dort verfügbar sind. Ein Beispiel dafür sind App Stores von Mobiltelefonen. Da der Aufwand für eine Plattform eine App zu entwickeln in etwa immer der selbe ist, werden App-Entwickler anhand der Marktposition eines App Stores agieren und so eventuell nur für die populärsten ihre App veröffentlichen.
		\paragraphlb{Wechselkosten}
		Eine weitere Ursache können Wechselkosten sein. Diese können technologisch, da man sich ein neues Produkt oder eine ganze Produktsparte kaufen muss um das neue System zu verwenden. Es kann jedoch auch wissensbezogen sein, da man beim Wechseln eventuell wieder eine Lernkurve hätte.
		\subsubsection{Gesetze der IKT-Branche}
		"Gesetze" im IKT sind keine niedergeschriebenen Gesetze, sondern Postulate, welche danach von der Industrie angenommen wurden.
		\paragraphlb{Moor'sches Gesetz}
		\paragraphlb{Metcalfes Gesetz}
		Der Wert eines Netzes steigt exponentiell mit der Menge an Nutzern
		\paragraphlb{Gilders Gesetz}
		Alle 18 Monate verdreifacht sich die Übertragungsbandbreite. Dieses Gesetz hat sich nicht im gleichen Ausmaß bewahrheitet.
		\paragraphlb{Huntleys Gesetz}
		Die Investition in Infrastrukturanlagen ist 10 Mal so teuer als die Investition in Produktionsanlagen. Da Infrastruktur viel öfter ausgetauscht werden muss, sind die langfristigen Kosten bedeutend höher als die von Produktionsstätten.
		\paragraphlb{Arthurs Gesetz}
		Alle diese Gesetze können in Arthurs Gesetz, als positives Rückkopplungsgesetz, zusammengefasst werden.
		\end{itemize}
		\subsubsection{Digitalisierung}
		Die Digitalisierung wird oft als Übertragung von analogen zu digitalen Medien verstanden. Digitalisierung sollte jedoch eine digitale Transformation eines analogen Prozesses sein. Wenn man zum Beispiel Netflix mit traditionellen Videoverleihen vergleicht, bieten diese zwar theoretisch den gleichen Service an. Ein Videoverleih ist jedoch physikalisch begrenzt: Es kann nur eine begrenzte Menge an Videos verfügbar sein, wodurch Videos eventuell durch neuere gewechselt werden. Ein digitales Geschäftsmodell ist jedoch inherent skalierbar. Jeder Nutzer auf Netflix kann ein Video gleichzeitig 'ausleihen' und es könnennn theoretisch alle Videos gleichzeitig zur Verfügung stehen. \\
		Die größten Vorteile der Digitalisierung sind, dass Daten und Informationen viel effizienter in digitaler Form übertragen werden können. Das ist jedoch auch gleichzeitig ein großer Nachteil, da so nicht nur die Quantität, sondern auch die Qualität sinkt. So erfährt man viel mehr über die Welt, erhält jedoch auch die Illusion, dass viel mehr passiert als früher, während man es sonst einfach nicht erfahren hat. Gleichzeitig sinkt jedoch auch die Qualität der veröffentlichten Information, da sich die Eintrittsbarriere, wer Information veröffentlichen kann, sinkt. \\
		\subsubsection{Globalisierung}
		Die Globalisierung ist der Effekt, dass die Welt untereinander besser erreichbar wird. Dadurch erhält man besseren Zugriff zu mehr potentiellen Kunden, jedoch auch mehr Konkurrenz von anderen Firmen. 
		\subsubsection{Vernetzung}
		Die Vernetzung erlaubt Kommunikation zwischen verschiedenen Benutzern. Diese Benutzer können Menschen, aber auch Maschinen sein. Bei der Kommunikation zwischen einem Menschen und einer Maschine, muss der Kontext berücksichtigt werden, da die Information welche von der Maschine bereitgestellt wird, vielleicht nur zu einer bestimmten Zeit relevant ist. Ein Teil der Vernetzung ist das Microtasking, bei welchem

	























  
\end{document}